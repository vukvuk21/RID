%-----------------------------------------------------------------------------
%   PACKAGES
%-----------------------------------------------------------------------------
\documentclass[12pt, a4paper, twocolumn]{article}
\usepackage[utf8]{inputenc}
\usepackage[T2A]{fontenc}
\usepackage[serbian]{babel}
\usepackage[margin=1in]{geometry} 
\usepackage{graphicx}
\usepackage{pgfplots}
\usepackage[backend=biber,
            natbib=true,
            url=false,
            doi=true,
            eprint=false
]{biblatex}

\pgfplotsset{compat=1.16}


%-----------------------------------------------------------------------------
%   TITLE PAGE
%-----------------------------------------------------------------------------
\title{Capture The Flag (CTF) kao uvod u računarsku bezbednost}
\author{Andrija Urošević, prof.\ dr.\ Sana Stojanović Đurđević\\Računarstvo i društvo\\Univerzitet u Beogradu\\Matematički fakultet}
\date{April, 2022.}


%-----------------------------------------------------------------------------
%   BIBLIOGRAPHY
%-----------------------------------------------------------------------------
\addbibresource{ctf_main.bib}


%-----------------------------------------------------------------------------
%   DOCUMENT
%-----------------------------------------------------------------------------
\begin{document}

\maketitle

%-----------------------------------------------------------------------------
%   ABSTRACT
%-----------------------------------------------------------------------------
\begin{abstract}
    Trenutno živimo u dobu koje iziskuje prikupljanje, obradu i deljenje 
    informacija. Informacije se najčešće čuvaju na računarima ili se dele
    putem Interneta. Svi ti računarski sistemi koji čuvaju informacije moraju
    biti zaštićeni od potencijalnog curenja informacija. Tada se u računarstvu
    javlja nova oblast koja se naziva računarska bezbednost. Zbog dinamike u 
    bavljenju računarskom bezbednošću, treniranje stručnih ljudi je veoma
    teško tradicionalnim edukacionim metodama. Jedan način rešavanja ovog 
    problema pruža CTF kao platforma za gejmifikovani proces učenja.

    \textbf{Ključne reči}: \emph{Capture The Flag}, \emph{CTF}, 
    \emph{računarska bezbednost}, \emph{internet bezbednost}, \emph{učenje}, 
    \emph{edukacija}.
\end{abstract}

%-----------------------------------------------------------------------------
\section{Uvod}
%-----------------------------------------------------------------------------

\emph{Gejmifikacija} je ideja koja koristi mehanike igara radi podsticanja i
motivisanja učesnika da se, kroz takmičarsko okruženje u kome je moguće 
pratiti napredak i relativno rangiranje, angažuju u aktivnosti u kojima je
stepen angažovanost nizak \cite{ctf_gami3}. Gejmifikacija ima široke primene
najviše u poslovanju i marketingu. Nedavno se pokazalo da gejmifikacija
ima primene i u procesu učenja. Da li je moguće iskoristiti CTF igru kao
platformu za gejmifikovani proces učenja o računarskoj bezbednosti?

Termin \emph{Capture The Flag} (CTF) se originalno odnosi na igru između dva
tima. Svaki tim ima zadatak da sačuva svoju fizičku zastavicu (\emph{flag}), 
dok u isto vreme pokušava da osvoji zastavicu (\emph{flag}) drugog tima.
Od 90-tih, format CTF-a se premešta na računare i Internet. 

\emph{Capture The Flag} (CTF) je takmičenje u oblasti računarske bezbednosti.
Cilj takmičenja je pronaći \emph{flag}-ove u nekom okruženju. Okruženje može
biti različitog opsega i formata, ali generalno pronađen \emph{flag} je dokaz
rešenog zadatka (npr.\ pristupljeno je skrivenim podacima ili bazi). Okruženje 
može biti jedan veb domen ili može biti mreža računara na većoj skali.

\emph{Flag} je tipično neki string karaktera, čiju specifikaciju daju
organizatori takmičenja. Specifikacija mora biti jasna svim učesnicima kako 
bi znali da su uspešno pronašli \emph{flag} i rešili zadatak. \emph{Flag}
obično ima neki prefiks koji učesnicima ukazuje da su zapravo pronašli 
\emph{flag}. Na primer, \emph{flag} može imati prefiks oblika \texttt{FLAG},
a sam \emph{flag} može izgledati kao \texttt{FLAG\{K73BSSxY3nFc1oAs9WwG\}}.
Prostor \emph{flag}-ova mora biti dovoljno velik, kako će ta činjenica 
sprečiti takmičare da koriste metod grube sile za pronalaženje \emph{flag}-a. 
Od takmičara se očekuje da pronađu \emph{flag} u datom okruženju.

Kada takmičar pronađe \emph{flag}, on ga šalje sistemu za verifikaciju, 
i ukoliko je \emph{flag} ispravan takmičar dobija poene za odgovarajući 
zadatak koji je rešio. Ovaj sistem je obično realizovan preko veb aplikacije, 
gde se takmičari mogu prijaviti, poslati pronađene \emph{flag}-ove, i uživo 
pratiti trenutne rezultate.

Nakon uspešno završenog takmičenja, javlja se potreba kod takmičara da podele
svoja genijalna rešenja sa ostalim učesnicima. Za osvojene \emph{flag}-ove
takmičari pišu iscrpe korake kako doći do tog \emph{flag}-a. Dokumenti koji 
nastaju nazivaju se \emph{writeup}-ovi. Preko \emph{writeup}-ova učesnici mogu 
da uporede svoja rešenja ili pronađu određeno rešenje za \emph{flag} koji 
nisu uspošno pronašli. U ovom procesu učesnici pospešuju svoja znanja, 
i stiču nove tehnike.

Postoje 3 glavne vrste CTF takmičenja: \emph{Jeopardy}, \emph{Attack-Defence},
i \emph{Mixed} \cite{ctf_time}. U \emph{Jeopardy} CTF-u, takmičari dobiju 
unapred zadatke, koji su zadati u statičkom okruženju, tj.\ drugi takmičari 
ne mogu uticati na okruženje i pri tome otežavati pronalaženje \emph{flag}-ova. 
\emph{Attack-Defence} stil podrazumeva borbu između više timova. Svaki tim 
ima sopstveno dinamičko okruženje u kome se nalaze \emph{flag}-ovi. Tim 
može menjati svoje dinamičko okruženje, i time osigurava svoj \emph{flag}. 
Ovaj postupak predstavlja odbranu (\emph{defence}) iz naziva. Napad 
(\emph{attack}) podrazumeva pronalaženje \emph{flag}-ova u protivničkom 
okruženju. Za svaku uspešnu odbranu od napada, ili uspešan napad timovi 
dobijaju \emph{flag} sa određenim brojem poena. \emph{Mixed} CTF može biti 
različitog formata, ali kao što ime sugeriše predstavlja mešavinu prethodna 
dva stila.

Popularnost CTF takmičenja raste iz godinu u godinu. \texttt{CTFTime} u
svojoj arhivi ima preko 300 javno dostupnih CTF takmičenja za 2021.\ godinu,
dok za 2016.\ godinu ima preko 100 javno dostupnih CTF takmičenja
\cite{ctf_time}. Jedno od prvih i najpoznatijih CTF takmičenja je DEFCON CTF
koji se svake godine održava na DEFCON konferenciji o računarskoj bezbednosti 
\cite{ctf_defcon}. Pored DEFCON CTF-a postoje i mnoga druga CTF takmičenja kao 
što su UCSB iCTF, Mozilla CTF, Facebook CTF, Google CTF, PHD CTF, RuCTFe, 
Hack.lu CTF, SECUINSIDE CTF, rwth CTF, CSAW CTF, PICO CTF \cite{ctf_rank}.
Jedan interesantan CTF, koji se održava u Srbiji svake godine, u okviru 
DESCON hakatona je DESCON CTF \cite{ctf_descon}. DESCON CTF je namenjen 
za početnike. Takođe, dobar resurs za vežbanje pre takmičenja predstavlja
\texttt{OverTheWire} \cite{ctf_otw}.

%-----------------------------------------------------------------------------
\section{Znanja i veštine koje se stiču kroz CTF}
%-----------------------------------------------------------------------------

Treniranje profesionalaca u oblasti računarske bezbednosti zahteva puno 
vremena i novca, ali pruža jedno veoma održivo globalno rešenje. Mnoge 
obrazovne institucije, društva informatičara, državne organizacije, i privatne
kompanije su svesne toga te konstantno uvode nove studijske programe, 
i kurseve. Jedan od tih studijskih programa je CSEC2017 \cite{ctf_csec}.

Pored formalnog obrazovanja, povećava se popularnost neformalnih metoda.
CTF predstavlja jednu takvu metodu gde učesnici poboljšavaju svoje znanje u
oblasti računarske bezbednosti kroz razne zadatke. Kako CTF zadaci često 
poseduju takmičarske elemente i elemente igre, oni su neformalnog karaktera
i teško je odrediti njihovu vezu sa formalnim metodama.

Postavlja se pitanje o tome kako i koliko su povezani formalni metodi,
u vidu studijskih programa, sa neformalnim metodama poput CTF-a. U daljem 
tekstu su opisane oblasti znanja koje definiše CSEC2017, nakon čega sledi 
studija o distribuciji tih oblasti znanja u CTF zadacima.

%-----------------------------------------------------------------------------
\subsection{CSEC2017 oblasti znanja}
%-----------------------------------------------------------------------------

CSEC2017 definiše osam oblasti znanja u računarskoj bezbednosti.
\begin{enumerate}
    \item \emph{Bezbednost podataka} sadrži kriptografiju, forenziku, 
        integritet podataka, i autentifikaciju.
    \item \emph{Bezbednost softvera} se fokusira na bezbednost u 
        programiranju, testiranju, i druge aspekte razvoja softvera.
    \item \emph{Bezbednost komponenti} se odnosi na bezbednosti komponenti 
        koje se integrišu u veće sisteme, što uključuje njihov dizajn i 
        obrnuto inženjerstvo.
    \item \emph{Bezbednost konekcije} podrazumeva mrežne servise, odbrane, i 
        napade.
    \item \emph{Bezbednost sistema} sadrži kontrolu pristupa, i etičko 
        hakovanje (eng.\ \emph{pen testing}).
    \item \emph{Bezbednost ljudi} se odnosi na čuvanje identiteta, podataka, 
        i privatnost. Sadrži socijalno inženjerstvo i svesnost o računarskoj 
        bezbednosti.
    \item \emph{Organizaciona bezbednost} se fokusira na menadžment rizika, 
        bezbednosne polise, i upravljanje incidentima na nivou organizacije.
    \item \emph{Društvena bezbednost} se bavi računarskom bezbednošću 
        na nacionalnom ili globalnom nivou.
\end{enumerate}

%-----------------------------------------------------------------------------
\subsection{Distribucija oblasti znanja u CTF zadacima}
%-----------------------------------------------------------------------------

Švábenský i dr.\cite{ctf_skills} ispitivali su distribuciju oblasti znanja u 
CTF zadacima. Ispitivanje je vršeno nad podacima, koji čine $5963$ 
\emph{writeup}-ova. Ovi podaci su preuzeti sa \url{CTFTime.org} koji u svojoj 
bazi, između ostalog, čuva i \emph{writeup}-ove raznih zadataka sa CTF 
takmičenja \cite{ctf_time}.

Metod podrazumeva pet faza. Prva faza je izdvajanje ključnih reči iz CSEC2017
\cite{ctf_csec}, koje određuju svako od znanja. Druga faza je preuzimanje i 
parsiranje \emph{writeup}-ova sa \url{CTFTime.org} \cite{ctf_time}. Treća faza 
predstavlja analizu \emph{writeup}-ova, tj.\ prebrojavanje instanci ključnih 
reči u svakom \emph{writeup}-u. Sledeća, četvrta faza predstavlja normalizaciju 
broja instanci ključnih reči. Poslednja, peta faza se sastoji u dodeljivanju 
\emph{writeup}-ova odgovarajućoj oblasti znanja \cite{ctf_skills}. 

Najzastupljenija oblast znanja za \emph{Jeopardy} stil CTF-a je 
\emph{bezbednost podataka}, dok \emph{bezbednost konekcije} i 
\emph{bezbednost sistema} zauzimaju drugo i treće mesto, respektivno. 
\emph{Bezbednost podataka} uključuje kriptografiju, i autentifikaciju, 
što opravdava prvo mesto zbog same prirode zadataka iz tih oblasti. Naime,
takvi zadaci su laki za dizajn i proveru. Za \emph{Attack-Defence} stil CTF-a
dobija se da je \emph{bezbednost konekcija} na prvom mestu. Ovaj rezultat ima
smisla kako timovi konstantno vrše napade na druge timove. Ostali rezultati 
se nalaze na slici~\ref{fig:ctf_ka} \cite{ctf_skills}. Interesantno je to da
dobijeni rezultati odgovaraju rezultatima o distribuciji oblasti znanja na 
master studijskim programima iz kurseva o računarskoj 
bezbednosti \cite{oth_ka, ctf_skills}.

\begin{figure}
    \begin{center}
        \begin{tikzpicture}[scale=0.6]
            \begin{axis}[
                    y=1cm, 
                    xbar, 
                    title={Distribucija oblasti znanja u CTF-u}, 
                    symbolic y coords={
                        Društvo, 
                        Organizacija, 
                        Ljudi, 
                        Sistemi, 
                        Konekcije, 
                        Komponente, 
                        Softver, 
                        Podaci
                    },
                    legend pos = south east, 
                    nodes near coords, 
                    xmax=50
                ]
                \addplot+ coordinates {
                    (2.96,Društvo) 
                    (9.86,Organizacija) 
                    (8.23,Ljudi) 
                    (12.72,Sistemi) 
                    (19.66,Konekcije) 
                    (8.94,Komponente) 
                    (10.02,Softver) 
                    (27.61,Podaci)
                }; 
                \addlegendentry{Jeopardy}
                \addplot+ coordinates {
                    (2.17,Društvo) 
                    (11.38,Organizacija) 
                    (9.18,Ljudi) 
                    (10.96,Sistemi) 
                    (32.68,Konekcije) 
                    (2.08,Komponente) 
                    (14.72,Softver) 
                    (16.83,Podaci)
                }; 
                \addlegendentry{Attack-Defence}
            \end{axis}
        \end{tikzpicture}
    \end{center}
    \caption{
        Distribucija oblasti znanja u $15 879$ \emph{jeopardy} i 
        $86$ \emph{attack-defense} \emph{writeup}-ova \cite{ctf_skills}.
    }\label{fig:ctf_ka}
\end{figure}

Glavno ograničenje ove analize je mali skup podataka nad kojima je analiza 
vršena, zajedno sa odbacivanjem polovine skupa podataka zbog neuspešnog 
parsiranja \cite{ctf_skills}. Postavlja se, takođe, pitanje o pouzdanosti 
samih \emph{writeup}-ova, tj.\ o njihovoj povezanosti sa samim CTF zadatkom. 
Pod pretpostavkom da \emph{writeup}-ove pišu entuzijasti i sami dizajneri CTF 
zadataka, možemo pretpostaviti njihovu pouzdanost.

%-----------------------------------------------------------------------------
\section{Problemi u CTF modelu}
%-----------------------------------------------------------------------------

Cilj CTF takmičenja je okupiti ljude koji se bave računarskom bezbednošću
različitog nivoa znanja i veština, kako bi oni mogli međusobno da dele
informacije, pored samog takmičarskog elementa. Čitava zajednica poštuje
CTF kao platformu za učenje novih veština, ali nekolicina članova direktno
priča o problemima sa kojima se susreću organizatori i takmičari, gde je svaki
problem detaljno opisan u narednim podsekcijama.

CTF se vrti oko računarske bezbednosti, ali je u suštini slobodan za bilo koje
druge oblasti. Zbog toga CTF zadatak može testirati dosta nepoznate oblasti 
znanja, koje se samo delimično mogu naći u literaturi, dokumentaciji, ili čuti 
na nekom kursu ili radionici. Dobra strana slobode koju CTF-ovi pružaju je u 
tome što otvara nove teme i uzdiže bezbednost na viši nivo. 

Sama struktura CTF zadataka se svodi na ``sve ili ništa'', tj.\ nije moguće
parcijalno rešiti neki zadatak. To otvara mogućnosti za dalje istraživanje, i
zajedno sa ograničenim vremenom forsira takmičare da budu efikasniji. Čak i
kada takmičar nije na pravom putu on može naučiti o nekoj oblasti. Ovo stvara
okruženje u kome najuporniji pobeđuju, bez obzira na njihov tehnički nivo.
Veruje se da ce uspešni takmičari biti oni koji se prepuste učenju i kulturi
CTF-ova.

Pored svih uspešnih stvari koje donosi CTF kao platforma za učenje i proveru
znanja, postoje problemi koji narušavaju ovaj model. Jedan od najvećih
problema je uvođenje novih ljudi u takmičenja \cite{ctf_chung}. Novi takmičari
počinju sa takmičenjem po preporuci. Oni veoma retko nastavljaju sa igranjem
CTF-a ukoliko ne uspevaju da reše ni jedan zadatak. Ovo može izazvati 
frustracije, te mnogi odustaju od takmičenja.

Sledeće podsekcije sadrže neke probleme koje su uočili Čang i Koen 
\cite{ctf_chung} u višegodišnjoj organizaciji CTF takmičenja.

%-----------------------------------------------------------------------------
\subsection{Težina igre}
%-----------------------------------------------------------------------------

CTF igre nisu jednostavne za igranje, tj.\ teško je ući u sam zadatak, čak i
za veterane, a pogotovu za nove takmičare. CTF zadatak objašnjava veoma
malo o tome kako ga rešiti, već prepušta igraču da sam smisli optimalan put
do rešenja, kao što je to u igrama poput šaha ili igre go. Baš na ovom aspektu
se dobija na mogućnosti izučavanja raznih oblasti. CTF podrazumeva da će
njegovi takmičari biti eksperti u svojoj oblasti, te iz godine u godinu
zajednica raste, a sami zadaci postaju sve teži. Ovo stvara veoma veliki 
pritisak na novog takmičara. Zbog toga je veoma teško početi sa takmičenjem,
čak i za one sa dobrim tehničkim predznanjem.

%-----------------------------------------------------------------------------
\subsection{Relacija između dizajna zadataka i njegove uspešnosti pri rešavanju}
%-----------------------------------------------------------------------------

Uspešan CTF događaj je onaj koji ima balans između učenja i provere znanja.
Zbog toga, organizatori imaju veoma težak posao da osmisle dovoljno teške 
zadatke iz kojih će takmičari steći znanje. Sa druge strane zadaci ne
smeju biti previše teški, jer se takmičari mogu zaglaviti na tom zadatku, i
pri tome se stvara rizik o izgubljenom vremenu učesnika i organizatora. Naime, 
takmičari neće pogledati ostale zadatke, i samim tim neće rešiti neki lakši 
zadatak i steći priliku za učenje nove oblasti. Pored toga, organizatori će
izgubiti vreme pri kreiranju zadataka koje niko neće pokušavati da reši. Iz
tog razloga dobri zadaci predstavljaju one zadatke koji navode do sopstvenog 
rešenja. Jedna metrika koja meri težinu zadatka može biti broj uspešnih 
rešenja. Lak zadatak će imati veliki broj uspešnih rešenja, dok će težak 
zadatak imati mali broj uspešnih rešenja.

Svakom zadatku organizatori dodeljuju poene, koje će takmičar dobiti pri
uspešnom rešavanju tog zadatka. Poeni su u direktnom odnosu sa težinom 
zadatka, pa će tako teški zadaci nositi puno poena, dok će laki zadaci nositi 
malo poena. Kako postoje mnogi CTF događaji, koji ciljaju na određene nivoe 
znanja, poeni predstavljaju lokalnu skalu težine zadataka tog CTF događaja.

Često organizatori pokušavaju da modifikuju zadatke, tako da oni postanu teži,
dodavanjem veštačkih konstrukcija. Ovakve konstrukcije su dizajnirane da 
namerno izazovu frustracije takmičara, i veruje se da će mali broj takmičara 
uspeti da ih reši. Ove tehnike otežavanja dovode do toga da zadatak ostane 
nerešen ili rešen od strane malog broja takmičara, jer zapravo uključuje 
faktor sreće. 

Još jedan način otežavanja zadatka je integrisanje grube sile u rešenje
zadatka. Gruba sila ne predstavlja dobar način savladavanja novih veština,
već samo dovodi do gubitka vremena pri rešavanju na CTF događaju gde je 
vreme veoma bitan faktor.

CTF takmičenja pored uobičajenih zadataka uključuju i veoma lake zadatke.
Laki zadaci podužu entuzijazam takmičara, i vagaju između igre i takmičenja. 
Oni su tu radi uživanja, jer će ih svaki takmičar rešiti bezmalo truda.

%-----------------------------------------------------------------------------
\subsection{Dokaz o kvalitetu}
%-----------------------------------------------------------------------------

Veterani su vešti u razumevanju postavke zadatka, ali problem nastaje kod 
novajlija. Naime, novajlije treba uvesti u novu oblast, kroz uvod u zadatak. 
Ne treba takmičare u potpunosti navoditi do rešenja, jer tu onda dolazi do 
slepog ispunjavanja uslova i gubi se na istraživanju i učenju. Kako bi ovaj 
problem bio smanjen, pri organizaciji i kreiranju zadataka najbolje je imati 
tim ljudi koji su na istom nivou kao i takmičari. Ovo želimo jer tada oni 
razmišljaju isto kao i takmičari, te mogu davati direktne povratne informacije 
o navođenju koje treba implementirati u zadatke. 

Nakon što je zadatak napravljen ulazi se u fazu dokaza o kvalitetu. Za
svaki CTF događaj ova faza je drugačija. Jedan primer može biti u tome
da organizatori između sebe dele zadatke i pokušavaju da reše tuđi zadatak.
Organizatori koji rešavaju zadatke direktno mogu da revidiraju zadatak, i
da daju povratne informacije o tome kako ga popraviti. Onda započinje novi
krug razvoja zadataka, koji se sastoji u poboljšanju trenutnih zadataka.

Mnogi CTF događaji zanemaruju ovu fazu, ili joj ne pridaju veliki značaj.
To dovodi do nerešivih zadataka, neadekvatno konfigurisane infrastrukture,
i zadataka čiji poeni ne reflektuju njihovu težinu.

%-----------------------------------------------------------------------------
\subsection{Poeni i njihov obrnuti efekat na takmičare i organizatore}
%-----------------------------------------------------------------------------

Dolazimo do interesantnog zapažanja nakon što takmičari dobiju zadatke.
Naime, takmičari će veštački odrediti težinu zadatka, pre čitanja uvoda u 
zadatak, samo na osnovu njegovog broja poena. Neiskusni takmičari će pomisliti
da nemaju dovoljno znanja za neki zadatak koji ima veliki broj poena, pa
će se fokusirati na one sa manjim brojem poena. Ova strategija odabira 
zadataka može dovesti do zaglavljanju na zadacima određene oblasti sa kojim
takmičar ima veoma malo iskustva. Sa druge strane, takmičar može imati
znanja o nekom zadatku sa više poena, te ga u potpunosti zanemariti, jer
nije uspeo rešiti onaj sa manjim brojem poena. Izbegavanje CTF zadataka
je trenutno nerešivi problem.

%-----------------------------------------------------------------------------
\subsection{Infrastruktura zadataka}
%-----------------------------------------------------------------------------

Interakcija sa takmičarima se obično odvija putom veb platforme. Na platformi 
se nalaze timovi, zadaci, sistem za bodovanje i proveru \emph{flag}-ova. U 
nekim situacijama dešava se korišćenje nefunkcionalnih veb platformi. Neki 
primeri: (1) nemogućnost učitavanja, (2) \emph{flag}-ove je moguće pronaći 
grubom silom, (3) takmičari ostvaruju proizvoljan broj poena. Ovo najčešće 
nastaje kada organizatori nemaju dovoljno vremena da osiguraju infrastrukturu.

Pre samog CTF događaja organizatori treba da testiraju sve moguće
propuste infrastrukture. Serveri treba da podnesu veliki saobraćaj.
Za svaki zadatak treba imati posebnu formu koja će proveravati \emph{flag},
zajedno sa vremenskim ograničenjem provere kako bi se sprečilo korišćenje
metoda grube sile. Treba očekivati varanja, i kao što se to očekuje na svim 
ostalim takmičenjima, te zbog toga treba osmisliti sisteme za detekciju 
varanja. Sve to doprinosi sigurnom i fer CTF događaju.

%-----------------------------------------------------------------------------
\subsection{Dvosmisleni zadaci}
%-----------------------------------------------------------------------------

Postoje primeri zadataka koji imaju više od jednog puta do rešenja. Isto
tako postoje zadaci kod kojih rešenja nisu jasno određena u opisu, i
sam tok rešavanja ne navodi do pravog rešenja. Ako spojimo ove dve činjenice
dobijamo dvosmislene zadatke koji se jedino mogu rešiti uz faktor sreće.
Ovakvi zadaci nastaju jer im se pridaje malo pažnje na kreativnoj formulaciji. 
Zbog toga svaki zadatak mora imati svoj opis koji pruža korisne informacije, 
zajedno sa navođenjem i uveravanjem takmičara da je na pravom putu.

%-----------------------------------------------------------------------------
\section{CTF na univerzitetskim kursevima}
%-----------------------------------------------------------------------------

Jedan od glavnih aspekata CTF takmičenja je u nadmudrivanju protivnika. Ali na
žalost ovaj aspekt nedostaje na mnogim univerzitetskim kursevima računarske
bezbednosti. Jedno rešenje predlažu Mirković i Piterson \cite{ctf_class} sa
svojom verzijom CTF-a: \emph{Class Capture The Flag} (CCTF). Naime, CCTF
zahteva minimalno dodatnog vremena od studenata, minimalno rada na 
osmišljanju zadataka od strane profesora i saradnika u nastavi, dok u isto
vreme uključuju studente u timski rad simulacijom realnog scenarija.
Svaki CCTF zadatak se fokusira na jednu oblast koja se obrađuje na datom 
kursu. Ovo doprinosi primeni znanja koje je student stekao tokom predavanja.
Nakon svakog CCTF događaja, vrši se analiza sa studentima o tome šta su
uradili ispravno, a šta pogrešno, sa ciljem unapređenja sledećeg CCTF 
događaja. 

%-----------------------------------------------------------------------------
\subsection{CCTF vs CTF}
%-----------------------------------------------------------------------------

CCTF se za razliku od tradicionalnog CTF-a održava na manjoj skali. Razlog 
toga je okruženje u kome se on izvršava. Naime, CCTF ima neke jedinstvene
osobine koje ga čine pogodnim za univerzitetske kurseve. Razlike između
CTF-a i CCTF-a su date na tabeli \ref{tab:cctf}. U nastavku su opisane neke 
osobine koje CCTF poseduje:

\begin{table}[!h]
    \centering
    \resizebox{0.7\textwidth}{!}{\begin{minipage}{\textwidth}
        \begin{tabular}{|c|c|c|}
            \hline
            \textbf{Osobina} & \textbf{CTF} & \textbf{CCTF} \\
            \hline
            \hline
            Pripremanje & nekoliko meseci & nekoliko nedelja \\
            Trajanje & 1-2 dana & 2 časa \\
            Uloge timova & crveni\footnote{Oni koji napadaju u \emph{Attack-Defense} CTF-u} ili plavi\footnote{Oni koji se brane u \emph{Attack-Defense} CTF-u} & crveni i plavi \\
            Uparivanje timova & svi na sve & parovi \\
            Učestalost & jednom godišnje & 2-3 puta po semestru \\
            Analiza & retko & uvek \\
            Težina & stručni & početni do srednji \\
            \hline
        \end{tabular}
        \end{minipage}}
    \caption{Upoređivanje CTF-a i CCTF-a.}\label{tab:cctf}
\end{table}

\begin{enumerate}
    \item \textbf{Laka realizacija.} CTF događaji zahtevaju najmanje 24 časa 
        za izvršavanje i mnogo nedelja za samu pripremu. Dok je za CCTF 
        potrebno oko dve nedelje priprema, i izvršava se u roku od 2 časa. 
        Time studenti i predavači ne gube vreme za ostale aktivnosti.
    \item \textbf{Učestaliji i sa analizom.} Mnogi CTF događaju se održavaju
        jednom godišnje i sa sobom nose malo pobednika i mnogo gubitnika.
        Ovo može da demotiviše one koji su izgubili i odvrati ih od oblasti
        računarske bezbednosti. Mnogi učesnici nemaju visok nivo edukacije,
        i iskustva. Oni zbog toga vrlo lako odustaju od samog takmičenja.
        Sa druge strane, CCTF se dešava više puta u toku jednog semestra.
        To omogućava studentima da se takmiče više puta godišnje sa minimalno
        dodatnog ulaganja i truda. Nakon svakog CCTF vrši se analiza, gde
        učesnici mogu diskutovati o strategijama koje su koristili, i tako
        unaprediti svoje veštine.
    \item \textbf{Dvostran.}
        CTF takmičenja su obično jednostrana, u smislu da timovi napadaju
        neki unapred određen sistem, ili da se brane od napada profesionalaca
        iz struke. CCTF su dvostrana, te će timovi biti u prilici da napadaju
        i da budu napadnuti. 
    \item \textbf{Doigravanje.}
        Za razliku od CTF takmičenja gde svaki tim igra protiv svih ostalih
        timova, CCTF uparuje timove u više iteracija. Ovo omogućava postojanje
        više pobedničkih timova tokom semestra i ima kao posledicu povećanje
        entuzijazma.
    \item \textbf{Svestran.}
        CCTF pruža mogućnost fokusiranja na određene oblasti, pa studenti
        mogu primeniti znanje koje su stekli tokom semestra. Sa druge strane,
        CCTF omogućava kombinaciju različitih oblasti, što će studente 
        naterati da posmatraju računarsku bezbednost u celosti. Takođe, 
        studenti će morati da donose brze odluke, i da snose posledice za one
        loše odabrane odluke.
\end{enumerate}

%-----------------------------------------------------------------------------
\subsection{Poboljšanje edukacije kroz CTF}
%-----------------------------------------------------------------------------

Gejmifikacija poboljšava pored ostalog i proces učenja \cite{gami1, gami2}.
Odatle dolazimo do prirodnog pitanja: Da li CTF (kao igra) može da poboljša
samo učenje o računarskoj bezbednosti. Postoji nekoliko radova koji se bave
ovom problematikom.

Leune i Petrilli \cite{ctf_leune} u svom radu postavljaju sledeće pitanje:
Da li se uključivanjem u realne simulacije odbrana i napada --- u obliku
CTF zadataka --- povećava efektivnost pri učenju o računarskoj bezbednosti?
Kao meru uspešnosti definišu sledeće hipoteze:

\begin{enumerate}
    \item Samopouzdanje studenata će se povećati učestvovanjem u CTF-u.
    \item Studenti će uživati u CTF-u.
    \item Studenti će steći praktične veštine učestvovanjem u CTF-u.
    \item Učestvovanje u CTF potkrepljuje teorijske koncepte.
\end{enumerate}

Prva hipoteza se pokazuje kao tačna. Mogućnost izvođenja raznih tehnika u 
kontrolisanom okruženju povećava samopouzdanje kod studenta da sam izvrši,
prepozna i odbrani se od napada. Druga hipoteza je, takođe, tačna. Mnogi
studenti su prijavili kako su veoma uživali u samom rešavanju zadataka, i
da su provodili mnogo više vremena na samim zadacima nego što su planirali.
I treća hipoteza je tačna, jer postoji jasna razlika između rezultata
onih koji su učestvovali u CTF-u i onih koji nisu. Četvrta hipoteza nije jasno
pokazana, ali se spekuliše da treba podeliti CTF zadatke na teorijske i
praktične, i da će ta podela doprineti poboljšanju samih rezultata kod
studenata.

%-----------------------------------------------------------------------------
\subsection{Prednosti i mane CTF zadataka na univerzitetskim kursevima}
%-----------------------------------------------------------------------------

CTF igre nisu idealno rešenje koje donosi samo dobre stvari sa sobom.
Kao i sve ostalo ima svoje mane. Neka zapažanja o dobrim i lošim stranama 
CTF zadataka na univerzitetskim kursevima nam daju Vikopal i dr.\ 
\cite{ctf_uni}:

\begin{enumerate}
    \item \textbf{Performanse studenata.} Postoji značajna statistička 
        korelacija između promenljivih koje su izvučene iz CTF zadataka i 
        rezultata kolokvijum/ispita. Korišćen je Spirmenov keoficijent 
        korelacije i dobijeni rezultati su u intervalu od $-0.5$ do $0.63$, 
        ali se veruje da su mnogo bolji zbog nedostatka vođenja evidencije
        o važnim događajima tokom CTF-a (vreme prikaza zadatka, vreme 
        podnošenja tačnog \emph{flag}-a, itd.).
    \item \textbf{Korisna navođenja.} Medijalna razlika između vremena
        pružanja smernica i podnošenja tačnog \emph{flag}-a pokazuje da su
        neke smernice korisnije od drugih. Treba izbegavati navođenja koja 
        su očigledna i ne pružaju nikakvu dodatnu informaciju kako bi se 
        problem rešio. Zbog toga ona moraju biti dovoljno informativna i 
        adaptivna, ali ne smeju biti u obliku uputstva.
    \item \textbf{Plagiranje CTF \emph{flag}-ova.} Primećeno je četiri vrste
        plagijata: (1) slanje istih \emph{flag}-ova u bliskim vremenskim
        opsezima; (2) korektan \emph{flag} je poslat kao nekorektan
        \emph{flag} na drugom zadatku: (3) zadaci su rešeni bez preuzimanja
        potrebnih fajlova; (4) brzo rešavanje uzastopnih zadataka.
    \item \textbf{CTF igre sa studentske strane.} Anketa koja je sprovedena 
        nakon uspešno završenog kursa pokazuje da većina studenata (13 od 16) 
        preferira CTF igre u odnosu na uobičajene zadatke koje imaju tokom 
        regularnih kurseva. Studenti koji preferiraju CTF igre, vole to što
        su igre zabavne, interaktivne, koriste razne alate, i uče kroz 
        pokušavanjem. Dva studenta su tvrdila da su CTF zadaci teži od 
        uobičajenih zadataka i da je potrebno puno vremena za njihovo 
        rešavanje. Jedan student je izjavio da je lakše prepisati CTF zadatak
        od uobičajenih zadataka (dovoljno je prepisati \emph{flag}).
\end{enumerate}

%-----------------------------------------------------------------------------
\section{Zaključak}
%-----------------------------------------------------------------------------

Videli smo da se formalni metodi učenja (kursevi, seminari, master 
studijski programi, itd.) i neformalni metodi učenja (CTF takmičenja) 
u velikoj meri poklapaju po oblastima znanja koje definiše CSEC2017. 
Naime, oba metoda pokazuju da se fokusiraju na određene oblasti 
znanja. Ovo ima smisla kako su baš te oblasti znanja primarne.

Dalje, uvideli smo da CTF takmičenja donose jako puno problema 
takmičarima kao i organizatorima takmičenja. Novi takmičari retko 
nastavljaju sa takmičenjima, jer su zadaci veoma teški. Organizatori 
veoma često u procesu otežavanja zadataka dodaju neke oblasti znanja 
koje su veoma šturo opisane u literaturi, ili dokumentaciji. Takođe, 
organizatori vrlo često zamrse problem tako da on postane nerešiv. 
Pored toga postoje i problemi sa infrastrukturom zadataka i platforme.

Na kraju smo razmatrali i pokušaje implementiranja CTF-a na 
univerzitetskim kursevima. Pored otklonljivih problema koje donosi 
ovakav poduhvat, ukupan rezultat je pozitivan i sa profesorske i sa 
studentske strane. 

\textbf{Dalja diskusija.} CTF donosi jako puno dobih osobina: razmišljanje 
van okvira poznatog, istraživanje, poboljšava efikasnost, timski rad, široko 
znanje iz mnogih oblasti, i mnoge druge intelektualne osobine. Pored toga 
donosi i nekakvo interno i društveno zadovoljstvo. Problem nastaje u tome što 
je CTF zajednica veoma zatvorena za početnike. Pored povećanja popularnosti 
potrebno je omogućiti novim članovima da se glatko upuste u CTF vode. CTF je 
igra čija pravila mogu biti definisana i adaptirana na mnoge oblasti van 
računarske bezbednosti. Te se postavlja pitanje: Da li je moguće jednako 
uspešno ili čak uspešnije primeniti CTF koncept na druge oblasti?

\nocite{*}

\printbibliography

\end{document}
