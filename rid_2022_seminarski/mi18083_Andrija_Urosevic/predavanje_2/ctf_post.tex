%-----------------------------------------------------------------------------
%	PACKAGES AND THEMES
%-----------------------------------------------------------------------------
\documentclass{beamer}
\usepackage[size=a3,orientation=portrait,scale=1.5]{beamerposter}
\usetheme{LLT-poster}
%\usecolortheme{ComingClean}
\usecolortheme{Entrepreneur}
%\usecolortheme{SimplePlus}
%\usecolortheme{ConspicuousCreep}  %% VERY garish.

\usepackage[utf8]{inputenc}
\usepackage[T1]{fontenc}
\usepackage{libertine}
\usepackage[scaled=0.92]{inconsolata}
\usepackage[libertine]{newtxmath}
\usepackage[serbian]{babel}
\usepackage{pgfplots}

%-----------------------------------------------------------------------------
%	TITLE AND AUTHOR
%-----------------------------------------------------------------------------

\author[andrija.urosevic@protonmail.com]{Andrija Urošević}
\title{Capture The Flag (CTF)\\  CTF kao uvod u računarsku bezbednost}
\institute{Univerzitet u Beogradu\\ Matematički fakultet}

%-----------------------------------------------------------------------------
%	POSTER
%-----------------------------------------------------------------------------
\begin{document}

\begin{frame}[fragile]
    \centering

%-----------------------------------------------------------------------------

    \begin{block}
        {Da li se CTF može koristiti kao platforma za gejmfikovani proces 
        učenja o računarskoj bezbednosti?}
        \begin{figure}
            \centering
            \includegraphics[width=0.8\textwidth]{Slike/ctf_gami.png}
            \caption{Gejmifikacija učenja računarske
            bezbednosti.}\label{fig:ctfgami}
        \end{figure}
    \end{block}

%-----------------------------------------------------------------------------
%   COLUMNS
%-----------------------------------------------------------------------------

    \begin{columns}[T]

%-----------------------------------------------------------------------------
%   FIRST COLUMNS
%-----------------------------------------------------------------------------
        \begin{column}{.46\textwidth}

%-----------------------------------------------------------------------------

            \begin{block}{CFT takmičenja}
                \begin{description}
                    \item [CTF] (Capture The Flag) je takmičenje u oblasti
                        računarske bezbednosti. Cilj takmičenja je pronaći
                        \emph{flag}-ove u nekom okruženju.
                    \item [Flag] je tipično neki string karaktera.
                \end{description}
            \end{block}

%-----------------------------------------------------------------------------

            \begin{block}{Znanja i veštine koje se stiču kroz CTF}
                \begin{figure}
                    \centering
                    \begin{tikzpicture}[scale=1]
                        \begin{axis}[
                            y=1cm,
                            xbar,
                            title={Distribucija oblasti znanja u CTF-u},
                            symbolic y coords={Društvo,
                                               Organizacija,
                                               Ljudi,
                                               Sistemi,
                                               Konekcije,
                                               Komponente,
                                               Softver,
                                               Podaci
                            },
                            legend pos = south east,
                            nodes near coords,
                            xmax=50
                            ]
                            \addplot+ coordinates {(2.96,Društvo)
                                                   (9.86,Organizacija)
                                                   (8.23,Ljudi)
                                                   (12.72,Sistemi)
                                                   (19.66,Konekcije)
                                                   (8.94,Komponente)
                                                   (10.02,Softver)
                                                   (27.61,Podaci)
                            };
                            \addlegendentry{Jeopardy}
                            \addplot+ coordinates {(2.17,Društvo)
                                                   (11.38,Organizacija)
                                                   (9.18,Ljudi)
                                                   (10.96,Sistemi)
                                                   (32.68,Konekcije)
                                                   (2.08,Komponente)
                                                   (14.72,Softver)
                                                   (16.83,Podaci)
                            };
                            \addlegendentry{Attack-Defence}
                        \end{axis}
                    \end{tikzpicture}
                    \centering
                    \caption{Distribucija oblasti znanja u $15 879$
                        \emph{jeopardy} i $86$ \emph{attack-defense}
                        \emph{writeup}-ova.}\label{fig:ctf_ka}
                \end{figure}
            \end{block}

%-----------------------------------------------------------------------------

        \end{column}

%-----------------------------------------------------------------------------
%   SECOND COLUMN
%-----------------------------------------------------------------------------
        \begin{column}{.46\textwidth}

%-----------------------------------------------------------------------------

            \begin{block}{Problemi u CTF modelu}
                \begin{itemize}
                    \item Težina igre
                    \item Relacija između dizajna zadataka i njegove
                        uspešnosti pri rešavanju
                    \item Dokaz o kvalitetu
                    \item Poeni i njihov obrnuti efekat na takmičare i
                        organizatore
                    \item Infrastruktura zadataka
                    \item Dvosmisleni zadaci
                \end{itemize}
            \end{block}

%-----------------------------------------------------------------------------

            \begin{block}{CTF na univerzitetskim kursevima}
                \begin{table}[!h]
                    \centering
                    \begin{tabular}{|c|c|c|}
                        \hline
                        \textbf{Osobina} & \textbf{CTF} & \textbf{CCTF} \\
                        \hline
                        \hline
                        Pripremanje & nekoliko meseci & nekoliko nedelja \\
                        Trajanje & 1-2 dana & 2 časa \\
                        Uloge timova & crveni ili plavi & crveni i plavi \\
                        Uparivanje timova & svi na sve & parovi \\
                        Učestalost & jednom godišnje & 2-3 puta po semestru \\
                        Analiza & retko & uvek \\
                        Težina & stručni & početni do srednji \\
                        \hline
                    \end{tabular}
                    \caption{Upoređivanje CTF-a i CCTF-a.}\label{tab:cctf}
                \end{table}
                \begin{itemize}
                    \item Samopouzdanje studenata će se povećati učestvovanjem
                        u CTF-u.
                    \item Studenti će uživati u CTF-u.
                    \item Studenti će steći praktične veštine učestvovanjem
                        u CTF-u.
                    \item Učestvovanje u CTF potkrepljuje teorijske koncepte.
                \end{itemize}
            \end{block}

%-----------------------------------------------------------------------------

        \end{column}

    \end{columns}

\end{frame}

\end{document}
