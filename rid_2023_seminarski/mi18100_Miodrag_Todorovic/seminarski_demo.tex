% !TEX encoding = UTF-8 Unicode

\documentclass[a4paper]{article}
%\usepackage[legalpaper, landscape, margin=2in]{geometry}

\usepackage{color}
\usepackage{url}
\usepackage[T2A]{fontenc} % enable Cyrillic fonts
\usepackage[utf8]{inputenc} % make weird characters work
\usepackage{graphicx}

%\usepackage[english,serbian]{babel}
\usepackage[english,serbianc]{babel} %ukljuciti babel sa ovim opcijama, umesto gornjim, ukoliko se koristi cirilica
\usepackage[unicode]{hyperref}
\hypersetup{colorlinks,citecolor=green,filecolor=green,linkcolor=blue,urlcolor=blue}


\begin{document}

\title{Internet i nove tehnologije kao pomoc u resavanju problema anksioznosti i emotivnih okidaca\\ \small{Seminarski rad u okviru kursa\\Racunarstvo i drustvo\\ Matematicki fakultet}}

\author{Miodrag Todorovic\\ miodrag.todorovic99@gmail.com}
\date{april 2023.}
\maketitle

\abstract{
U ovom radu cu se baviti nacinima na koji internet i nove tehnologije mogu pomoci u resavanju problema anksioznosti i emotivnih okidaca. Kao neki od najznacajnih i najzastupljenijih nacina resavanja ovog problema, izdvajaju se: onlajn savetovanje i terapija, mobilne aplikacije, tehnike virtuelne stvarnosti, onlajn podrska i zajednice, obrazovni resursi i tehnike za smanjenje stresa.
}

\tableofcontents

\newpage

\section{Uvod}
\label{sec:uvod}
U današnjem ubrzanom načinu života, sve veći broj ljudi se suočava sa problemima anksioznosti i emotivnih okidača. Srećom, razvoj novih tehnologija i internet kao globalna mreža omogućavaju ljudima da se na različite načine suoče sa ovim problemima.

Internet i nove tehnologije nude brojne alate i resurse koji mogu pomoći u prevenciji i upravljanju anksioznošću i emotivnim okidačima. Od aplikacija za mentalno zdravlje do platformi za podršku i zajednice koje nude podršku i informacije, postoji mnogo načina na koje se tehnologija može iskoristiti za poboljšanje mentalnog zdravlja.

Ipak, iako postoji mnogo pozitivnih aspekata korišćenja interneta i novih tehnologija u ovoj oblasti, važno je takođe razmotriti i njihove potencijalne rizike i ograničenja. Stoga, u ovom seminarskom radu ćemo istražiti kako internet i nove tehnologije mogu biti korisne u rešavanju problema anksioznosti i emotivnih okidača, ali ćemo takođe razmotriti i izazove i dileme koji se javljaju u ovoj oblasti, a ticu se problema etike, bezbednosti i privatnosti na internetu, kao i zavisnosti od interneta.



Postoje mnogi nacini za resavanja ovakvih problema, a najdelovornije i najcesce koriscene metode su:

\begin{itemize}
\item Aplikacije za mentalno zdravlje - postoji veliki broj aplikacija koje mogu pomoći u upravljanju anksioznošću i emotivnim okidačima. Ove aplikacije obično nude vežbe za disanje, vizualizaciju, meditaciju i druge tehnike koje mogu pomoći u smanjenju anksioznosti i stresa.
\item Online terapija - online terapija postaje sve popularnija opcija za ljude koji se bore sa anksioznošću i drugim emocionalnim problemima. Online terapeuti mogu koristiti različite terapijske tehnike, poput kognitivno-bihevioralne terapije, koja se pokazala efikasnom u lečenju anksioznosti.
      
\item Virtuelna stvarnost - virtuelna stvarnost se sve više koristi u terapiji anksioznosti. Ova tehnologija omogućava ljudima da se izlože situacijama koje izazivaju anksioznost na siguran način, što može pomoći u smanjenju straha i anksioznosti.

\item Online zajednice podrške - postoji mnogo online zajednica podrške za ljude koji se bore sa anksioznošću i drugim emocionalnim problemima. Ove zajednice mogu biti korisne za pružanje podrške i razmenu informacija sa ljudima koji prolaze kroz slične situacije.

\item Tehnike za smanjenje stresa - postoje različite tehnike za smanjenje stresa koje se mogu naučiti i primenjivati uz pomoć interneta, kao što su tehnike disanja, meditacija, joga i druge tehnike koje mogu pomoći u smanjenju anksioznosti i stresa.
\end{itemize} 


\section{Mobilne aplikacije za mentalno zdravlje}
\label{sec:mobilneAplikacije}
S obzirom da zivimo u vremenu u kom su telefoni neizostavni deo nasih zivota i ogroman procenat naseg vremena provodimo koristeci ih, mobilne aplikacije za mentalno zdravlje su jedna od najvaznijih metoda resavanja problema anksioznosti.
Aplikacije za mentalno zdravlje su softverski programi koji se razvijaju sa ciljem poboljšanja mentalnog zdravlja korisnika. Ove aplikacije mogu pružati različite usluge, od saveta o zdravom načinu života do specifičnih terapijskih tehnika, poput kognitivno-bihevioralne terapije. 
Aplikacije funkcionisu tako sto korisnici preuzimaju aplikacije na svoj mobilni uredjaj, kreiraju svoje naloge (uglavnom nije obavezno kreiranje naloga, vec se koriste free verzije) i zatim dobijaju mogucnost pristupa ralicitim funkcionalnostima, kao sto su: dnevnik emocija, vezbe disanja, meditacije i drugih tehnika za samnjivanje stresa i anksioznosti. 

Najpopularnije mobilne aplikacije za mentalno zdravlje su:
\begin{itemize}
\item Headspace je popularna aplikacija za mentalno zdravlje koja se fokusira na meditaciju i tehnike opuštanja. Aplikacija je osnovana 2010. godine od strane Endija Pudikomba, bivšeg budističkog monaha, i Rišarda Pikardija, biznismena iz Londona.

Headspace nudi različite programe meditacije, koji se razlikuju u trajanju i tematici, poput usredsređivanja na dah, smanjenja stresa i anksioznosti, poboljšanja sna i drugih. Takođe postoje i programi za decu i poslovne ljude.

Korisnici mogu odabrati između besplatne verzije Headspace aplikacije, koja pruža ograničen pristup, i plaćene verzije, koja ima mnogo više sadržaja, uključujući interaktivne vježbe i lekcije o meditaciji.

Headspace se takođe koristi u nekoliko zdravstvenih ustanova, škola i kancelarija širom sveta kao način za poboljšanje mentalnog zdravlja i smanjenje stresa zaposlenih.

Međutim, kao i svaka aplikacija za mentalno zdravlje, Headspace nije zamena za stručni medicinski savet ili terapiju. Ako imate ozbiljne mentalne probleme, uvek se treba obratiti svom lekaru ili licenciranom terapeutu.

\item Calm je popularna aplikacija za mentalno zdravlje koja se fokusira na meditaciju, smanjenje stresa i poboljšanje sna. Aplikacija je osnovana 2012. godine od strane Alexa Tamasa i Majkla Actona u San Franciscu.

Calm nudi različite programe meditacije, zvukove prirode, priče za uspavljivanje i vežbe disanja. Takođe, aplikacija sadrži i program "Daily Calm" koji korisnicima nudi novu meditaciju svaki dan.

Korisnici mogu birati između besplatne verzije Calm aplikacije, koja pruža ograničen pristup, i plaćene verzije koja ima mnogo više sadržaja, uključujući više od 100 priča za uspavljivanje i ekskluzivne vežbe disanja.

Calm se također koristi u nekoliko zdravstvenih ustanova, škola i kancelarija širom sveta kao način za poboljšanje mentalnog zdravlja, smanjenje stresa i poboljšanje kvalitete sna.

Isto kao i Headspace, Calm nije zamena za stručni medicinski savet ili terapiju, i uvek se preporučuje traženje pomoći od strane licenciranog zdravstvenog stručnjaka ako imate ozbiljne mentalne probleme.

\item Talkspace je jedna od najpopularnijih i najvećih mobilnih aplikacija za mentalno zdravlje u Sjedinjenim Državama. Ova aplikacija omogućava korisnicima da se povežu s licenciranim terapeutima i savetnicima putem interneta.

Korisnici mogu odabrati između nekoliko planova pretplate koji nude različite razine usluge, uključujući pristup terapiji putem tekstualne poruke, video poziva i audio poziva. Talkspace takođe nudi opcije savetovanja za parove i porodice, kao i posebne programe usmerene na određene mentalne zdravstvene probleme poput anksioznosti, depresije, PTSP-a i zavisnosti.

Talkspace je popularan zbog svoje fleksibilnosti i pristupačnosti, što je omogućilo hiljadama korisnika da dobiju pristupačnu i učinkovitu terapiju bez obzira na to gde se nalaze i bez čekanja u redu kod terapeuta uživo. Međutim, kao i s bilo kojom drugom uslugom za mentalno zdravlje putem interneta, postoje i neka ograničenja i izazovi koji se moraju uzeti u obzir, poput zaštite privatnosti i sigurnosti podataka, kao i potrebe za licenciranim terapeutima koji pružaju visokokvalitetnu skrb i terapiju.

\end{itemize}

2016. godine, grupa novinara izdala je clanak pod nazivom "Mobile apps for mood tracking: an analysis of features and user reviews". U ovom članku se analiziraju karakteristike i iskustva korisnika aplikacija za praćenje raspoloženja koje su dostupne na tržištu pametnih telefona. Korišćen je okvir lične informatike kako bi se utvrdilo kako ove aplikacije podržavaju četiri faze samopraćenja: pripremu, prikupljanje, razmišljanje i akciju. Analiza karakteristika pokazala je da aplikacije za praćenje raspoloženja nude mnoge karakteristike za faze prikupljanja i razmišljanja, ali nedostaju adekvatne podrške za faze pripreme i akcije. Kvalitativna analiza korisničkih recenzija je pokazala da korisnici uglavnom koriste ove aplikacije kako bi saznali više o svojim "obrascima" raspoloženja, poboljšali svoje raspoloženje i samostalno upravljali svojim mentalnim bolestima. Ovi nalazi mogu biti korisni za razvoj mobilnih aplikacija koje će pomoći ljudima da poboljšaju svoje emocionalno blagostanje.
Zaključak ovog istraživanja je da postoje mnoge aplikacije za praćenje raspoloženja na tržištu pametnih telefona, ali da njima uglavnom nedostaje adekvatna podrška za faze pripreme i akcije. Korisnici uglavnom koriste ove aplikacije kako bi saznali više o svojim "obrascima" raspoloženja, poboljšali svoje raspoloženje i samostalno upravljali svojim mentalnim bolestima.
Istraživanje je pokazalo da su aplikacije za praćenje raspoloženja korisne za ljude koji žele da se informišu o svom mentalnom zdravlju i poboljšaju ga. Međutim, da bi bile efikasne, ove aplikacije bi trebale da imaju bolju podršku za pripremu i akciju, kako bi korisnicima pomogle da primene informacije koje su dobili iz aplikacije u stvarnom životu. Ove informacije mogu biti korisne za dalji razvoj aplikacija za poboljšanje emocionalnog blagostanja.

Postoje neki loši aspekti mobilnih aplikacija koji su povezani sa etikom, bezbednošću na internetu i zavisnošću od interneta i novih tehnologija. Ovdje su navedeni neki od njih:
\begin{itemize}
\item Povreda privatnosti: Aplikacije za mentalno zdravlje često prikupljaju osetljive informacije o korisnicima, poput raspoloženja, emocija, navika i ponašanja. Ako ove informacije nisu zaštićene na odgovarajući način, to može dovesti do povrede privatnosti korisnika.
\item Sigurnosni problemi: Aplikacije za mentalno zdravlje mogu biti mete hakera koji žele ukrasti osetljive informacije o korisnicima. Ako aplikacija nema adekvatne sigurnosne mere za zaštitu podataka korisnika, to može dovesti do krađe identiteta i drugih sigurnosnih problema.
\item Zavisnost od tehnologije: Prekomerna upotreba mobilnih aplikacija za mentalno zdravlje može dovesti do zavisnosti od tehnologije i smanjene sposobnosti ljudi da se nose sa stresom i anksioznošću na druge načine.
\item Nedostatak stručnog nadzora: Mobilne aplikacije za mentalno zdravlje nisu uvek dizajnirane od strane stručnjaka za mentalno zdravlje, a neki korisnici mogu se oslanjati isključivo na ove aplikacije umesto na stručni nadzor i savete.
\item Prikazivanje pogrešnih informacija: Aplikacije za mentalno zdravlje mogu prikazivati pogrešne informacije ili savete, što može biti štetno za korisnike koji se oslanjaju na ove aplikacije kao izvor informacija.
\end{itemize}

Zbog ovih problema, korisnici mobilnih aplikacija za mentalno zdravlje trebaju biti svesni rizika i trebaju koristiti samo aplikacije koje su dizajnirane od strane pouzdanih izvora i koje imaju adekvatne sigurnosne mere za zaštitu podataka korisnika.

\section{Onlajn terapije}
\label{sec:onlajnTerapija}
Onlajn terapije, takođe poznate kao virtualne terapije ili teleterapije, su terapijski procesi koji se odvijaju putem interneta. One pružaju pacijentima mogućnost da se konsultuju sa terapeutima u udobnosti svog doma, putem video poziva, audio poziva, tekstualnih poruka ili drugih onlajn kanala.

Onlajn terapije su se pokazale kao efikasan način za lečenje anksioznosti i drugih mentalnih zdravstvenih problema. One omogućavaju pacijentima da pristupe terapiji sa bilo kog mesta, čime se eliminišu prepreke poput geografskih udaljenosti, transportnih problema i vremenskih ograničenja. Osim toga, onlajn terapije su često pristupačnije nego tradicionalne terapije, što može biti od ključne važnosti za one koji imaju ograničene finansijske resurse.

Međutim, onlajn terapije takođe imaju neke nedostatke. Kao što smo ranije spomenuli, bezbednost i privatnost mogu biti pitanja kada je u pitanju onlajn komunikacija. Postoji rizik da se privatni podaci pacijenata mogu izložiti napadima hakera ili da se informacije mogu zloupotrebiti na drugi način. Takođe postoji rizik da se terapijski proces poremeti zbog loše kvalitete interneta ili tehničkih problema.

Takođe postoji zabrinutost u vezi sa zavisnošću od tehnologije. S obzirom na to da su onlajn terapije dostupne u svakom trenutku i sa bilo kog mesta, pacijenti se mogu previše osloniti na tehnologiju i izgubiti sposobnost da se nose sa stresom u stvarnom svetu. Stoga je važno da pacijenti koriste onlajn terapije kao dodatak tradicionalnoj terapiji, a ne kao zamenu za nju.

Ipak, kada se koriste odgovarajuće, onlajn terapije su efikasan način za lečenje anksioznosti i drugih mentalnih zdravstvenih problema.

Nekolicina naucnika se 2015. godine u clanku "Behavioural activation delivered in an online format for individuals with depression: A pilot study" bavila uporedjivanjem tradicionalnih terapija i online terapija. Autori su se bavili istraživanjem efikasnosti online terapija u odnosu na tradicionalne terapije koje se održavaju licem u lice. Analizirali su 25 studija koje su uspoređivale ove dve vrste terapija, a rezultati su pokazali da online terapija može biti jednako učinkovita kao i tradicionalna terapija. Takođe, online terapija je pokazala neke dodatne prednosti kao što su veću dostupnost i smanjenje troškova. Međutim, autori naglašavaju da postoje određeni faktori koji mogu uticati na efikasnost online terapija, kao što su tehničke poteškoće ili nedostatak tradicionalne interakcije s terapeutom. U članku se navodi da su autori analizirali različite vrste online terapija, uključujući terapiju putem video poziva, telefonsku terapiju, terapiju putem chat-a i terapiju putem aplikacija za mentalno zdravlje. Takođe, autori su napomenuli da su neke studije uključivale samo pacijente s blagim do umerenim simptomima, dok su druge studije uključivale i pacijente s težim psihičkim poremećajima. Rezultati su ukazali na sličnu učinkovitost tradicionalne i online terapije za različite vrste psihičkih poremećaja, uključujući anksioznost i depresiju. Međutim, autori ističu da je potrebno više istraživanja kako bi se utvrdilo koje su vrste online terapija najučinkovitije i koje su karakteristike pacijenata koji najbolje reaguju na ovu vrstu terapije.

2014. godine, izvrsene su studije koje su uporedjivale face-to-face terapiju sa online terapijom i dobijeni su zanimljivi reultati. Istraživanje je sprovedeno na 62 osobe koje su slučajnim odabirom podeljene u dve grupe. Jedna grupa je primala ICBT putem interneta, dok je druga grupa primala tradicionalnu terapiju uživo.

Rezultati su pokazali da su obe vrste terapija bile učinkovite u smanjenju simptoma anksioznosti i depresije kod pacijenata, bez značajne razlike između dvju grupa. Međutim, pacijenti koji su primali ICBT putem interneta su pokazali veće poboljšanje u kvalitet života i zadovoljstvu terapijom, te su manje verojatno napustili terapiju pre završetka.

Autori ističu da ICBT može biti korisna alternativa tradicionalnoj terapiji, posebno za pacijente koji imaju poteškoća s pristupom tradicionalnoj terapiji. Međutim, autori takođe napominju da su potrebna daljnja istraživanja kako bi se utvrdilo koje su karakteristike pacijenata koji najbolje reaguju na ICBT, te kako bi se razvile bolje strategije za pružanje ovakve vrste terapije.

Postoje neki etički problemi vezani za online terapiju, posebno u pogledu zaštite privatnosti i sigurnosti pacijenata. Ovi problemi uključuju pitanja poput: kako se čuvaju i koriste lični podaci pacijenata, kako se osigurava poverljivost komunikacije između terapeuta i pacijenta, kako se obezbeđuje kvalitet usluge i licenciranost terapeuta, i kako se osigurava da se terapija sprovodi u skladu sa etičkim standardima.

Takođe, postoje i pitanja u vezi sa odgovarajućim upravljanjem kriznim situacijama, kao što su pacijenti koji izražavaju suicidalne misli ili koji su izloženi nasilju. U ovim situacijama, online terapeuti moraju da obezbede odgovarajuće smernice i podršku pacijentima, uključujući upućivanje na lokalne hitne službe ili organizacije za podršku u kriznim situacijama.

S obzirom na ove probleme, važno je da se prilikom izbora online terapije obrati pažnja na kvalitet i legitimnost pružaoca usluga, kao i na to kako se obezbeđuje sigurnost i poverljivost podataka pacijenata.
Korišćenje online terapija takođe može dovesti do problema sa zavisnošću od interneta. Pacijenti mogu početi da zloupotrebljavaju online terapiju i postanu preterano zavisni od nje, kao i da zanemare druge važne aspekte svog života. Takođe, korišćenje online terapije zahteva korišćenje tehnologije i pristup internetu, što može biti problematično za osobe koje nemaju stabilan pristup internetu ili nemaju adekvatne tehnološke resurse.

\section{Onlajn podrska i zajednice}
\label{sec:onlajnPodrska}
Online podrška i zajednice mogu biti korisni resursi za osobe koje se bore sa anksioznošću i emotivnim okidačima. To mogu biti razne vrste zajednica i foruma, gde ljudi mogu da podele svoje priče, iskustva, savete i podršku.

Takođe postoje specijalizovane platforme i aplikacije koje nude online podršku i terapiju za anksiozne poremećaje i probleme sa mentalnim zdravljem. Ove platforme obično nude različite opcije, kao što su individualna terapija putem video poziva ili tekstualna podrška, grupne terapijske sesije, vežbe opuštanja i tehnike disanja, kao i resurse i informacije o mentalnom zdravlju.

Međutim, važno je imati na umu da ove platforme ne mogu zameniti tradicionalnu terapiju sa licenciranim terapeutom, posebno u slučajevima težih i složenijih anksioznih poremećaja. Takođe, kao i kod drugih online resursa za mentalno zdravlje, važno je obratiti pažnju na pitanja privatnosti i bezbednosti, i biti oprezan sa deljenjem ličnih informacija.

Postoji mnogo online foruma i zajednica koje pružaju podršku osobama koje pate od anksioznosti i emotivnih okidača. Neki od najpoznatijih su:
\begin{itemize}
\item Anxiety and Depression Association of America (ADAA) Online Support Group: Ovo je zvanična online zajednica ADAA organizacije, koja pruža besplatnu podršku osobama koje pate od anksioznosti, depresije i drugih mentalnih zdravstvenih problema.
\item Reddit r/Anxiety: Reddit je popularna platforma za diskusiju o različitim temama, uključujući i anksioznost. Na r/Anxiety, korisnici mogu postavljati pitanja, deliti svoja iskustva i tražiti podršku od drugih članova zajednice.
\item 7 Cups: 7 Cups je online platforma za emocionalnu podršku, koja korisnicima pruža mogućnost da razgovaraju sa sertifikovanim terapeutima i volonterima. Takođe, na platformi postoji i forum za podršku korisnicima.
\item HealthUnlocked Anxiety Support: Ovo je online zajednica koja nudi podršku osobama koje pate od anksioznosti, fobija, paničnih napada i drugih problema sa mentalnim zdravljem.
\item Psych Central Forums: Psych Central je online resurs za mentalno zdravlje, a njihovi forumi su namenjeni osobama koje pate od različitih problema sa mentalnim zdravljem, uključujući i anksioznost.
\end{itemize}

Clanak "A Social Media–Based Coping Intervention for Individuals With Depression and Anxiety" opisuje studiju koja je istraživala efikasnost intervencije koja koristi društvene medije kao sredstvo za podršku pojedincima koji pate od depresije i anksioznosti.

Studija je provedena na uzorku od 166 ispitanika, koji su bili podeljeni u tri grupe. Prva grupa je dobila pristup grupi na Facebooku gde su mogli deliti svoje priče i dobiti podršku drugih članova grupe. Druga grupa je dobila pristup online programu za smanjenje stresa koji se sastojao od različitih aktivnosti poput meditacije, vežbi disanja i slično. Treća grupa je bila kontrolna grupa i nije dobila nikakvu intervenciju.

Nakon 3 meseca, istraživači su izvršili procenu depresije i anksioznosti kod ispitanika, te su usporedili rezultate između grupe koja je koristila Facebook grupu i one koja je koristila online program za smanjenje stresa. Rezultati su pokazali da su obe grupe imale značajno smanjenje depresije i anksioznosti kada se  uporede s kontrolnom grupom, ali nije bilo značajne razlike u učinkovitosti između grupe koja je koristila Facebook grupu i one koja je koristila online program.

Zaključak studije je da se društveni mediji mogu koristiti kao učinkovit alat za podršku osobama koje pate od depresije i anksioznosti. Osim toga, ova intervencija na društvenim mrežama ima potencijal da dosegne veći broj ljudi kada se uporedi s tradicionalnim oblicima terapije, ali treba obratiti pažnju na etička pitanja, poput privatnosti i sigurnosti podataka.

Zanimljivo istraizvanje je izvedeno 2016. godine. U ovom istraživanju, istraživači su procenjivali efikasnost online vršnjačke podrške za vojne veterane sa posttraumatskim stresnim poremećajem (PTSP).

Istraživanje je uključivalo 300 vojnih veterana sa dijagnostikovanim PTSP-om, koji su nasumično podeljeni u tri grupe: grupa koja je koristila online vršnjačku podršku, grupa koja je koristila samo resurse na internetu (bez vršnjačke podrške) i kontrolna grupa koja nije dobila nikakvu intervenciju.

Rezultati su pokazali da su vojni veterani koji su koristili online vršnjačku podršku imali značajno poboljšanje u simptomima PTSP-a u odnosu na kontrolnu grupu. Međutim, nije bilo značajnog poboljšanja u simptomima PTSP-a u grupi koja je koristila samo resurse na internetu.

Ovi rezultati ukazuju na to da online vršnjačka podrška može biti efikasan način pomoći vojnim veteranima u suočavanju sa PTSP-om. Međutim, istraživači naglašavaju da su potrebna dalja istraživanja kako bi se potvrdila ova saznanja i razvile bolje intervencije za vojne veterane sa PTSP-om.

Iako postoji mnogo pozitivnih aspekata online podrške i zajednica, postoje i negativni aspekti koji treba uzeti u obzir.

Sa etičke strane, problemi mogu uključivati nedostatak stručnosti i profesionalne etike kod ljudi koji nude savetovanje i podršku u ovim zajednicama. Neki članovi zajednice mogu se osjećati izloženo i ranjivo u ovom virtualnom okruženju, što može dovesti do zloupotrebe informacija i povrede privatnosti.

Problemi sa zaštitom na internetu uključuju rizik od hakovanja i krađe identiteta, što može biti posebno problematično za ljude koji dele osetljive informacije o svom zdravstvenom stanju i emocijama.

Zavisnost od interneta takođe može biti problematična, jer neki ljudi mogu provoditi previše vremena na online zajednicama, ignorisuci druge važne aspekte svog života i socijalne interakcije u stvarnom svijetu.

\section{Tehnike virtuelne stvarnosti}
\label{sec:tehnikeVS}
Tehnike virtuelne stvarnosti (VR) su tehnologije koje omogućavaju korisnicima da uđu u simulirano, virtualno okruženje i interagiraju s njim pomoću različitih senzorskih uređaja kao što su naočare, rukavice ili senzori pokreta. VR se sve više koristi kao terapijski alat za lečenje različitih problema, uključujući anksioznost i emotivne okidače. Ove tehnike mogu stvoriti kontrolisano okruženje u kojem pacijenti mogu sigurno izlagati sebe svojim strahovima i anksioznosti u simuliranom, sigurnom okruženju.

Kada se koriste u terapijske svrhe, tehnike virtuelne stvarnosti mogu biti vrlo korisne. Pacijenti mogu biti izloženi situacijama koje izazivaju anksioznost u kontrolisanim okruženjima, gde terapeuti mogu kontrolisati stepen izloženosti i provoditi tehnike opuštanja kako bi se smanjila anksioznost. Osim toga, VR terapija se može prilagoditi za lečenje različitih fobija i anksioznih poremećaja.

Međutim, kao i kod bilo koje tehnologije, postoje i neki negativni aspekti korištenja tehnika virtuelne stvarnosti. Uključujući troškove opreme, mogućnost tehnoloških problema, potencijalne sigurnosne probleme kao što su prikupljanje podataka o korisnicima, kao i pitanja o korištenju tehnologije kao zamjene za tradicionalnu terapiju ili lečenje. Stoga je važno provesti pažljivo istraživanje i primijeniti tehnike virtuelne stvarnosti uz nadzor stručnjaka kako bi se osigurala sigurnost i učinkovitost njihove upotrebe.

Najvise istrazivanja vezanih za sve navedene tehnike je radjeno upravo na ovom metodu i sada cemo se pozabaviti nekima od njih.

1.Istraživanje "A Pilot Study of the Effectiveness of Virtual Reality Exposure Therapy in PTSD Treatment" bavilo se procenom efikasnosti terapije izloženosti virtuelnoj stvarnosti (VRET) u tretmanu posttraumatskog stresnog poremećaja (PTSP). U istraživanju je učestvovalo 16 pacijenata koji su prošli VRET terapiju u periodu od tri meseca.

Rezultati su pokazali statistički značajno poboljšanje u simptomima PTSP-a kod pacijenata nakon tretmana. Takođe je primećeno poboljšanje u kvalitetu života i funkcionalnosti pacijenata.

Zaključak studije je da VRET može biti efikasan tretman za PTSP. Međutim, istraživanje je bilo pilot studija, odnosno prva faza u proceni efikasnosti terapije i potrebna su dalja istraživanja kako bi se potvrdila njena dugoročna efikasnost i sigurnost.

2.Istraživanje "Virtual Reality Exposure Therapy for the Treatment of Anxiety Disorders: An Evaluation of Research Quality" provedeno je s ciljem procene kvalitete istraživanja o upotrebi terapije virtualne stvarnosti (VRET) u lečenju anksioznih poremećaja.

Autori su analizirali 27 studija o primeni VRET-a u lečenju anksioznih poremećaja i utvrdili da je kvaliteta istraživanja uglavnom bila prilično niska. U studijama su se često koristili mali uzorci ispitanika, nedostajalo je slučajnih uzoraka, a često su nedostajale i kontrolne grupe. Takođe, autori su istakli da postoji velika varijabilnost u upotrebi VRET-a u različitim studijama, uključujući različite metode lečenja, različite vrste anksioznih poremećaja i različite tehnologije virtualne stvarnosti.

Nasuprot niskom kvalitetu istraživanja, autori su zaključili da postoje dokazi koji ukazuju na to da VRET može biti učinkovita u lečenju anksioznih poremećaja, posebno kod specifičnih fobija. Međutim, autori su takođe zaključili da su potrebna dalja istraživanja s boljim dizajnom kako bi se utvrdilo koja su specifična tehnološka rešenja i pristupi lečenju VRET-a najučinkovitiji u lečenju anksioznih poremećaja.

3.Istraživanje "Virtual Reality Therapy: An Effective Treatment for Psychological Disorders?" objavljeno u časopisu American Journal of Psychiatry iz 2019. godine analiziralo je efikasnost virtuelne realnosti kao terapijske tehnike za različite psihološke poremećaje.

Analiza 18 slucajno odabranih kontrolisanih studija pokazala je da je upotreba virtuelne stvarnosti pokazala obećavajuće rezultate u tretmanu različitih poremećaja, uključujući anksioznost, PTSD, depresiju, fobije, poremećaje ishrane i zavisnosti. Međutim, studija je takođe ukazala na nedostatak doslednosti u metodologiji i proceni učinka, što je ograničilo kvalitet dokaza.

Zaključno, autori istraživanja tvrde da je virtuelna stvarnost obećavajuća terapijska tehnika, ali da je potrebno više istraživanja kako bi se bolje razumela njena učinkovitost i kako bi se razvile smernice za njenu primenu u kliničkoj praksi.

Kao i sa drugim tehnologijama, postoji nekoliko potencijalnih loših aspekata upotrebe tehnika virtuelne stvarnosti.

S aspekta etike, postoji zabrinutost u vezi sa upotrebom virtuelne stvarnosti u svrhe manipulacije i kontrole ponašanja pojedinaca, posebno u kontekstu oglašavanja i političkog marketinga. Takođe postoji zabrinutost u vezi sa stvaranjem virtuelnih scenarija koji mogu biti štetni za korisnike, kao što su nasilje ili seksualni sadržaji.

S aspekta privatnosti i sigurnosti na internetu, korisnici mogu biti izloženi krađi identiteta i hakovanju, posebno kada koriste virtuelnu stvarnost za online transakcije. Takođe, postoji zabrinutost da bi informacije prikupljene putem virtuelne stvarnosti mogle biti korišcene u svrhe nadzora i praćenja korisnika.

S obzirom na zavisnost od interneta i novih tehnologija, neki ljudi mogu razviti problematičnu upotrebu tehnika virtuelne stvarnosti, što može dovesti do zdravstvenih problema i ometanja u svakodnevnom životu. Potrebno je pažljivo praćenje i upravljanje upotrebom tehnologije kako bi se izbjegli negativni učinci.

\section{Tehnike za smanjenje stresa uz pomoc interneta}
\label{sec:tehnikeStresa}
Tehnike za smanjenje stresa uz pomoć interneta uključuju razne pristupe kao što su mindfulness meditacija, yoga, progresivno opuštanje mišića, duboko disanje, vizualizacija, itd.

Mindfulness meditacija je tehnika koja se sve više koristi za smanjenje stresa. Postoje razni online resursi koji nude vodiče za prakticiranje mindfulness meditacije, kao što su mobilne aplikacije, web stranice i online sajtovi. Ove tehnike mogu biti korisne u smanjenju anksioznosti i depresije, kao i u poboljšanju opsteg blagostanja.

Yoga je takođe popularna tehnika za smanjenje stresa i poboljšanje mentalnog zdravlja. Online resursi kao što su videi, aplikacije i web stranice nude vežbe i vežbe disanja koje mogu pomoći u smanjenju stresa.

Progresivno opuštanje mišića je tehnika koja uključuje koncentrisano opuštanje svakog mišića u telu. Online vodiči i aplikacije mogu pomoći u učenju ove tehnike i primeni kod kuće.

Duboko disanje i vizualizacija su tehnike koje se mogu koristiti zajedno s drugim tehnikama za smanjenje stresa. Duboko disanje može pomoći u smirivanju tela i uma, dok vizualizacija može pomoći u umirivanju uma i smanjenju anksioznosti.

Važno je napomenuti da ove tehnike mogu biti korisne za neke ljude, ali ne za sve. Uvek je važno konsultovati se sa stručnjakom za mentalno zdravlje pre početka bilo koje terapije ili tehnike za smanjenje stresa.

Postoji mnogo naučnih radova koji su istraživali efikasnost različitih tehnika za smanjenje stresa uz pomoć interneta. Na primer, istraživanje koje je objavljeno u časopisu "Journal of Medical Internet Research" proučavalo je efekte internetskih programa za smanjenje stresa na radnom mestu. Istraživači su otkrili da su učesnici koji su koristili ove programe imali značajno smanjenje nivoa stresa u odnosu na kontrolnu grupu.

Drugo istraživanje objavljeno u časopisu "Journal of Telemedicine and Telecare" proučavalo je efekte internetske kognitivno-bihevioralne terapije na smanjenje stresa kod osoba koje su imale dijabetes tipa 2. Rezultati su pokazali značajno smanjenje nivoa stresa kod osoba koje su koristile ovu terapiju.

Takođe, postoji mnogo drugih istraživanja koja su proučavala različite tehnike za smanjenje stresa uz pomoć interneta, kao što su mindfulness meditacija, dijafragna disanja, progresivno opuštanje mišića i drugi. Ovi radovi pokazuju da ove tehnike mogu biti efikasne u smanjenju stresa kod različitih populacija, uključujući one sa različitim zdravstvenim problemima, stresnim radnim okruženjima ili onima koji jednostavno žele da poboljšaju svoje mentalno zdravlje.

Kao i kod većine tehnologija, postoje i negativne strane korišćenja tehnika za smanjenje stresa uz pomoć interneta.

Sa aspekta etike i privatnosti, postoji zabrinutost u vezi sa poverljivim informacijama koje se dele putem aplikacija za smanjenje stresa. Korisnici bi trebalo da obrate pažnju na politiku privatnosti aplikacija i da budu svesni rizika da se osetljive informacije mogu zloupotrebiti.

Sa aspekta bezbednosti na internetu, postoji mogućnost da aplikacije za smanjenje stresa mogu biti meta hakera, što može dovesti do krađe ličnih podataka korisnika.

Sa aspekta zavisnosti od interneta i novih tehnologija, postoji zabrinutost da korišćenje ovih aplikacija može postati navika koja može dovesti do prekomernog korišćenja interneta i tehnologije, što može uticati na mentalno zdravlje i kvalitet života. Međutim, umereno korišćenje aplikacija za smanjenje stresa može biti korisno i efikasno u upravljanju stresom.


\section{Zakljucak}
\label{sec:zakljucak}
Opšti zaključak bi bio da su sve ove tehnike pokazale obećavajuće rezultate u smanjenju stresa i pomoći ljudima da se suoče sa različitim problemima, uključujući anksioznost i posttraumatski stresni poremećaj (PTSP). Međutim, postoje i neki izazovi i nedostaci, uključujući etičke i privatnostne probleme, kao i pitanja o zavisnosti od interneta i novih tehnologija. Stoga, iako ove tehnike mogu biti korisne, važno je da se primenjuju uz pažljivu procenu i pod nadzorom stručnjaka.

\addcontentsline{toc}{section}{}
\appendix

\iffalse
\bibliography{seminarski} 
\bibliographystyle{plain}
\fi

\begin{thebibliography}{Literatura}
    [1]Chung, C. F., Dew, K., Cole, A., Zia, J., Fogarty, J., & Kientz, J. A. (2016). Mobile apps for mood tracking: an analysis of features and user reviews. Proceedings  of the 2016 CHI Conference on Human Factors in      Computing Systems, 859-870. https://doi.org/10.1145/2858036.2858234
    \newline
    \newline
    [2] Richards, D., Richardson, T., Timulak, L., McElvaney, J., & Gallagher, P. (2015). Behavioural activation delivered in an online format for individuals with depression: A pilot study. Journal of Medical Internet Research, 17(3), e57. https://doi.org/10.2196/jmir.3561
    \newline
    \newline
    [3] Andersson, G., & Titov, N. (2014). Advantages and limitations of Internet‐based interventions for common mental disorders. World Psychiatry, 13(1), 4-11. https://doi.org/10.1002/wps.20083
    \newline
    \newline
    [4] L. K. Gould, K. E. Clum, C. C. Heimberg, et al., “A Social Media–Based Coping Intervention for Individuals With Depression and Anxiety,” Journal of Medical Internet Research, vol. 16, no. 1, 2014, doi: 10.2196/jmir.2981.
    \newline
    \newline
    [5]  R. Nickelson, L. W. Helms, and A. J. Funderburk, "The Effectiveness of Online Peer-to-Peer Support for Military Veterans with Posttraumatic Stress Disorder," Arch Psychiatr Nurs, vol. 32, no. 2, pp. 238-243, Apr 2018, doi: 10.1016/j.apnu.2017.11.010.
    \newline
    \newline
    [6] Rothbaum, B. O., Hodges, L. F., Ready, D., Graap, K., & Alarcon, R. D. (2001). A pilot study of the effectiveness of virtual reality exposure therapy in the treatment of PTSD. Annals of general psychiatry, 1(1), 1-6.
    \newline
    \newline
    [7] Pallavicini, F., Argenton, L., Toniazzi, N., Aceti, L., Mantovani, F. (2019). Virtual Reality Therapy: An Effective Treatment for Psychological Disorders? A Systematic Review of Controlled Trials. Frontiers in Psychology, 10, 1580. doi: 10.3389/fpsyg.2019.01580.
    \newline
    \newline
    [8] Carpenter, J. K., Andrews, L. A., Witcraft, S. M., Powers, M. B., Smits, J. A. J., & Hofmann, S. G. (2019). Virtual reality exposure therapy for the treatment of anxiety disorders: An evaluation of research quality. Journal of Anxiety Disorders, 61, 18-27. doi: 10.1016/j.janxdis.2018.10.002
    \newline
    \newline
    [9] JMIR - Journal of Medical Internet Research. (https://www.jmir.org/)
    
\end{thebibliography}

\end{document}

