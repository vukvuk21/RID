% !TEX encoding = UTF-8 Unicode

\documentclass[a4paper]{article}

\usepackage{color}
\usepackage{url}
\usepackage[T2A]{fontenc} % enable Cyrillic fonts
\usepackage[utf8]{inputenc} % make weird characters work
\usepackage{graphicx}
\usepackage{amsmath, derivative}

\usepackage[english,serbian]{babel}
\DeclareUnicodeCharacter{0301}{*************************************}

\usepackage[unicode]{hyperref}
\hypersetup{colorlinks,citecolor=green,filecolor=green,linkcolor=blue,urlcolor=blue}

\newtheorem{primer}{Primer}[section]

\begin{document}

    \begin{titlepage} % Suppresses headers and footers on the title page

	\centering % Centre everything on the title page
	
	\scshape % Use small caps for all text on the title page
	
	\vspace*{\baselineskip} % White space at the top of the page
	
	%------------------------------------------------
	%	Title
	%------------------------------------------------
	
	\rule{\textwidth}{1.6pt}\vspace*{-\baselineskip}\vspace*{2pt} % Thick horizontal rule
	\rule{\textwidth}{0.4pt} % Thin horizontal rule
	
	\vspace{0.75\baselineskip} % Whitespace above the title
	
	{\LARGE Anksioznost\\ i\\ emotivni okidači\\} % Title
	
	\vspace{0.75\baselineskip} % Whitespace below the title
	
	\rule{\textwidth}{0.4pt}\vspace*{-\baselineskip}\vspace{3.2pt} % Thin horizontal rule
	\rule{\textwidth}{1.6pt} % Thick horizontal rule
	
	\vspace{2\baselineskip} % Whitespace after the title block
	
	%------------------------------------------------
	%	Subtitle
	%------------------------------------------------
	
	Seminarski rad u okviru kursa Računarstvo i društvo \\ profesor: dr Sana Stojanović Djurdjević % Subtitle or further description
	
	\vspace*{3\baselineskip} % Whitespace under the subtitle
	
	%------------------------------------------------
	%	Editor(s)
	%------------------------------------------------
	
	
	
	\vspace{0.5\baselineskip} % Whitespace before the editors
	
	{\scshape\Large Jelena Bondžić 131/2018 \\} % Editor list
	
	\vspace{0.5\baselineskip} % Whitespace below the editor list
	
	\textit{Matematički fakultet \\ Univerziteta u Beogradu} % Editor affiliation
	
	\vfill % Whitespace between editor names and publisher logo
	
	%------------------------------------------------
	%	Publisher
	%------------------------------------------------
	
	
	\vspace{0.3\baselineskip} % Whitespace under the publisher logo
	
	april 2023. % Publication year
	

\end{titlepage}
 

	
\newpage	
    
\tableofcontents
		
\newpage
		
\section{Uvod}
\label{sec:uvod}
	Tehnološki napredak promenio je način na koji živimo, radimo i komuniciramo sa drugim ljudima. Većina ljudi ima pristup mnoštvu tehnoloških uređaja i platformi koje nam omogućavaju da ostanemo povezani sa svetom oko nas. Iako je tehnologija na mnogo načina olakšala naše živote, ona je takođe donela nove izazove koji mogu mnogo da utiču na anksioznost. Anksioznost je uobičajeno stanje koje pogađa milione ljudi širom sveta, a  upotreba tehnologije na loš način je identifikovana kao značajan faktor u razvoju anksioznosti \cite{prva}. Kada govorimo o emocionalnim okidačima, prvenstveno mislimo na događaje ili stimulanse koji izazivaju jak emocionalni odgovor kod pojedinaca, a poznato je da upotreba tehnologije izaziva emocionalne reakcije kod ljudi. Cilj ovog rada je da podigne svest o uticaju tehnologije na mentalno zdravlje ljudi. \\
 
     \textbf{Anksioznost} se približnije definiše kao psihološko stanje koje karakteriše osećaj straha, zabrinutosti i nervoze. Može se ispoljiti različitim fizičkim simptomima kao što su znojenje, ubrzan rad srca, kratak dah, stezanje u grudima, nervozan stomak... Anksioznost je prirodan odgovor na stres i može biti korisna u malim dozama, jer može pomoći da se osoba motiviše za akciju. Međutim, kada anksioznost postane prekomerna ili hronična, ona može ometati svakodnevni život i imati negativan uticaj na mentalno zdravlje i u tom slučaju već govorimo o patološkoj anksioznosti. Anksioznost je složeno stanje koje može imati više uzroka, uključujući genetske faktore i životna iskustva. Za pojedince koji se bore sa anksioznošću, emotivni okidači mogu biti posebno izazovni jer mogu izazvati iznenadne i intenzivne simptome anksioznosti. \\
    
    Anksioznost postaje sve prisutnija tokom godina, sa sve više i više ljudi koji prijavljuju simptome ovog stanja. Porast anksioznosti tokom godina doveo je do većeg interesovanja u istraživanju ovog stanja, kao i do razvoja novih pristupa lečenju. 
    
      
              
                
                 Neki od novijih podataka \cite{treca} pokazuju da:
                \begin{itemize}
    		 	\item Mlađi ljudi češće imaju neki oblik anksioznosti.
    		 	\item Tokom 2021. godine, 28\% mladih osoba od 16 do 29 godina su najverovatnije već imale neki oblik anksioznosti
    		 	\item Više je žena koje su doživele teži oblik anksioznosti od muškaraca. U 2022. i 2023. godini u proseku 37,1\% žena i 29,9\% muškaraca prijavilo je visok nivo anksioznosti 
    		 	\item Tokom pandemije COVID-19 došlo je do povećanja broja ljudi koji su prijavili visok nivo anksioznosti, ali je broj počeo da se smanjuje od tada
                    \item Preko 80\% studenata se suočava sa anksioznošću \cite{druga}
                   
    		 
    		 			 	
    		 \end{itemize}  
   		

    \section{Etika i privatnost na internetu}
    \label{sec:podnaslov2}
        Upotreba tehnologije je stvorila nova etička razmatranja u oblasti mentalnog zdravlja i anksioznosti. Na primer, uvodjenje virtuelne stvarnosti i drugih tehnologija u terapiji lečenja anksioznosti je pokrenula etička pitanja u vezi sa sigurnošću ovih pristupa. Pored toga, upotreba veštačke inteligencije i algoritama mašinskog učenja u dijagnostici i lečenju mentalnog zdravlja izaziva etičku zabrinutost u vezi sa privatnošću i potencijalnom zloupotrebom ličnih podataka. Kako tehnologija nastavlja da napreduje, važno je razmotriti etičke aspekte njene upotrebe u kontekstu mentalnog zdravlja i anksioznosti. \\
        
        Široka upotreba društvenih mreža i onlajn komunikacije izazvala je zabrinutost za privatnost, bezbednost podataka i maltretiranje putem interneta, što može da pogorša anksioznost i izazove emocionalni stres. Pitanje privatnosti na internetu je posebno važno u kontekstu anksioznosti, jer pojedinci mogu oklevati da podele lične podatke ili traže pomoć iz straha od izloženosti ili osude. U kontekstu ovih problema, počinje da raste svest o odgovornom razvoju tehnologije kada je reč o mentalnom zdravlju. Ovo uključuje razvoj tehnologija za očuvanje privatnosti, kao što su bezbedne platforme za razmenu poruka i šifrovano skladištenje podataka, kao i promovisanje transparentnosti i kontrole korisnika nad ličnim podacima. \\
        
        Kako tehnologija nastavlja da igra sve važniju ulogu u mentalnom zdravlju, ključno je dati prioritet privatnosti i etičkim pitanjima kako bi se osiguralo da su prava i dobrobit pojedinaca zaštićeni.

              \section{Cenzura i sloboda govora}
             \label{sec:naslovCenzura}

             Zabrinutost zbog cenzure i ograničavanja slobode govora takođe se pojavila u kontekstu anksioznosti i mentalnog zdravlja. Na primer, neke platforme društvenih mreža su primenile politike za uklanjanje postova i naloga koji se odnose na samopovređivanje i samoubistvo, za koje neki tvrde da ograničavaju pristup pojedincima ključnim informacijama i resursima. S druge strane, neki tvrde da su takve politike neophodne da bi se sprečilo širenje štetnog sadržaja i zaštitili ugroženi pojedinci. \\
            
             Štaviše, zabrinutost oko cenzure se proteže izvan teme mentalnog zdravlja i uključuje politička i društvena pitanja. Kako su tehnološke platforme postale sve uticajnije u oblikovanju javnog mišljenja, pitanja o cenzuri i slobodi govora su postala složenija. Neki tvrde da cenzura određenih stavova ili perspektiva može narušiti osnovna prava pojedinaca na slobodu govora, dok drugi tvrde da platforme imaju odgovornost da ograniče širenje dezinformacija i govora mržnje.\\

             Jedan od primera cenzure može se videti u Severnoj Koreji, gde vlasti strogo kontrolišu pristup internetu i sprovode sistemsku cenzuru informacija. Severnokorejski režim primenjuje rigorozne mere kako bi ograničio pristup spoljnim informacijama i zadržao kontrolu nad informacijama koje se šire unutar zemlje. Ovo uključuje filtriranje i blokiranje odredjenih veb stranica, socijalnih mreža i drugih online platformi koje nisu u skladu sa trenutnim političkim režimom. Ljudima je ograničeno znanje o svetu van granica zemlje, a nepoštovanje pravila može imati ozbiljne posledice, uključujući pritvor i druge kanznene mere. Ovakav oblik cenzure ograničava mogućnosti građana da steknu različite perspektive, obrazuju se i slobodno izraze svoje mišljenje, što može doprineti osećaju izolacije i povećanju anksioznosti u takvom društvu.

        
    \section{Zavisnost od interneta}
    \label{sec:podnaslov3}
        Zavisnost od interneta je oblik bihevioralne zavisnosti koju karakteriše preterano i opsesivno korišćenje interneta, što često dovodi do negativnih posledica kao što su društvena izolacija, smanjena produktivnost i poremećeni obrasci spavanja. Istraživanja sugerišu da zavisnost od interneta može biti rasprostranjenija među osobama sa anksioznim poremećajima, jer je verovatnije da će koristiti internet kao odbrambeni mehanizam za suočavanje sa simptomima anksioznosti \cite{cetvrta}.
        Prekomerna upotreba interneta takođe može dovesti do povećane anksioznosti i emocionalnog stresa. Na primer, upotreba društvenih mreža je povezana sa povećanim poređenjem sa drugim ljudima, strahom od propuštanja informacija i iskustava, kao i sa niskim samopoštovanjem, što može izazvati anksioznost i druge probleme mentalnog zdravlja. Štaviše, opsesivna provera e-pošte, sajtova sa vestima i drugih onlajn platformi može dovesti do osećaja preopterećenosti, što može pogoršati simptome anksioznosti. \\ \newline
        Neke od statistika o zavisnosti od društvenih mreža \cite{peta} :
            \begin{itemize}
		 	\item Očekuje se da će do 2025. godine mesečni broj aktivnih korisnika društvenih mreža dostići 4,41 milijardu, što je otprilike jedna trećina celokupne svetske populacije.
		 	\item Procenjuje se da više od 200 miliona ljudi širom sveta pati od zavisnosti od društvenih mreža i interneta.
		 	\item Simptomi depresije se dvostruko češće pojavljuju kod tinejdžera koji provode više od 5 sati dnevno na svojim pametnim telefonima.
		 	\item Statistika zavisnosti od društvenih mreža otkriva da 15\% ljudi starosti od 23 do 38 godina priznaje da je zavisno od društvenih mreža.
		 	\item Proveravanje društvenih mreža tokom vožnje dešava se kod 55\% vozača.
            \end{itemize}  

            \section{Socijalna anksioznost i društvene mreže}
		\label{sec:podnaslov4}
	
	         Socijalna anksioznost (socijalni anksiozni poremećaj) podrazumeva intenzivni strah od različitih društvenih situacija, posebno situacije koje nisu poznate ili u kojoj osoba oseća da će biti posmatrana ili vrednovana od strane drugih. Osnova socijalne anksioznosti je strah od
              toga da ćete biti proučavani, osuđivani ili 	\begin{figure}[h]
            \caption{Socijalna anksioznost i društvene mreže}
            \centering
            \includegraphics[scale=0.2]{social-media-2.eps}
            \end{figure} osramoćeni u javnosti. \\ 
    
            Uspon platformi društvenih mreža transformisao je način na koji se povezujemo i komuniciramo sa drugima. Međutim, uticaj korišćenja društvenih mreža na pojedince sa socijalnom anksioznošću izazvao je zabrinutost u vezi sa potencijalnim efektima na njihovo mentalno stanje. Prekomerna upotreba društvenih mreža može pojačati osećaj socijalne anksioznosti. Jedan od ključnih faktora je stalna izloženost pažljivo odabranim i idealizovanim verzijama života drugih ljudi, što može izazvati društvena poređenja i osećaj nedovoljnosti. Razvoj socijalne anksioznosti može biti podstaknut kod onih mladih koji stavljaju znak jednakosti između društvenih mreža i ,,ideala’’ kao što su popularnost, uspeh i prihvaćenost. To znači da oni broj lajkova, komentara i pratilaca vide kao izvor ličnog vrednovanja i na osnovu istih grade svoj identitet. Strah od propuštanja (FOMO) je još jedan aspekt koji utiče na pojedince sa socijalnom anksioznošću, jer mogu iskusiti anksioznost kada vide druge da se bave društvenim aktivnostima iz kojih se osećaju isključenima. \\
        
            Šest važnih pitanja koja treba da se zapitate ako osećate da ste zaglavljeni u krugu zavisnosti od društvenih mreža iz koje ne možete da pobegnete su:
    		\begin{enumerate}
			\item
			Da li provodite mnogo vremena kada niste na mrežama razmišljajući o društvenim mrežama ili planiranju da koristite društvene mreže?
			\item
			Da li osećate potrebu da vremenom sve više koristite društvene mreže?
			\item
			Da li koristite društvene mreže da biste zaboravili na lične probleme?
			\item
			Da li često pokušavate da smanjite upotrebu društvenih mreža, ali bezuspešno?

			\item
			Da li postajete anksiozni ili uznemireni ako ne možete da koristite društvene mreže?
			\item
			Da li toliko koristite društvene mreže da je to imalo negativan uticaj na vaš posao, vezu ili studije?
			
		\end{enumerate}

            \begin{figure}[h]
            \caption{ Uticaj društvenih mreža na tinejdžere \cite{peta}}
            \centering
            \includegraphics[scale=0.4]{socialMedia.eps}
            \end{figure}
            \\
		
		\section{Tehnologija kao pomoć kod anksioznosti}	
		\label{sec:termini_i_citiranje}
            Kako anksiozni poremećaji pogađaju milione pojedinaca širom sveta, informisanje o korisnim aspektima tehnologije je ključno, tj. kako tehnologija može pružiti podršku, resurse i alate koji pomažu pojedincima da se nose sa anksioznošću u svakodnevnom životu. 


            
              \par Tehnologija omogućava sesije terapije na daljinu, tako da se terapijama sa stručnjacima pristupa putem video poziva. Ovo je posebno pogodno osobama sa anksioznošću kojima je izazov da odu na terapiju uživo. 
              \par  Dostupne su različite aplikacije i digitalni alati koji nude vežbe opuštanja, vođene meditacije i tehnike disanja. Takođe su mnoge od aplikacija povezane sa psiholozima i psihoterapeutima kojima možete da se obratite kroz ovaj vid komunikacije, a pored toga nude i edukativni sadržaj o mentalnom zdravlju. Ovakve aplikacije mogu pomoći u upravljanju simptomima anksioznosti. Neke od tih aplikacija su: Headspace, Calm, Happify, MindShift CBT - Anxiety Relief, Edupression.
              \par  Tehnologija omogućava lak pristup obrazovnim resursima o anksioznosti, njenim uzrocima i smernicama upravljanja. Online članci, blogovi, video snimci i podkasti nude informacije i smernice, osnažujući osobe sa anksioznošću da saznaju o svom stanju i istraže više o tome.
              \par Tehnologija nudi niz opcija za zabavu i smanjenje stresa koje mogu pomoći da odvratite pažnju i opustite se. Igranje igrica, slušanje muzike, gledanje filmova ili TV emisija ili učešće u kreativnim projektima može da obezbedi preko potreban predah od anksioznih misli i podstakne opuštanje.

               \par Zajednica sa drugima koji dele slična iskustva može smanjiti osećaj izolacije i pružiti osećaj pripadnosti i validacije. Ovakve online zajednice neguju empatiju, razumevanje i ohrabrenje, stvarajući prostor gde pojedinci mogu da traže podršku i dele strategije suočavanja sa problemom koji imaju.
               \par Tehnologiju treba koristiti umereno i ograničiti vreme provedeno ispred ekrana. Uvek je bolje što više vremena provoditi u prirodi, radeći neku aktivnost, baviti se sportom, ostvarivati komunikaciju uživo i na taj način doživljavati nova iskustva, ali naravno u svemu tome treba naći mesto za upotrebu tehnologije na pametan i zrdrav način tako da doprinosi našem telu i umu.
         
            \section{Zanimljivosti: prva svedočenja o anksioznim poremećajima}
		\label{sec:podnaslov4}
		
	
  			
            \begin{figure}[h]
            \caption{ Anksioznost u Antici }
            \centering
            \includegraphics[scale=0.5]{Greek-cure-for-depression-and-anxiety-1.eps}
            \end{figure}
              U zapadnoj literaturi, najstariji opis simptoma anksiozne grupe poremećaja, nalazi se u Homerovoj Ilijadi napisanoj oko 720. godine pre nove ere. U indijskoj literaturi se pominje oko 5000. godine pre nove ere u Ramajani, iako nije opisan kao anksioznost ili bilo kojim drugim sličnim imenom \cite{sesta}.
            U doba Antičke Grčke (776–323. pre nove ere) je skovan termin „histerija“ \cite{sedma}. Reč „histerija“ zapravo ima svoj koren u grčkom srodnom terminu za matericu, „histera“, najverovatnije zato što su ljudi u to vreme verovali da utiče samo na žene. Tokom rane Renesanse (15. vek), žene koje su bile veoma anksiozne i sklone „histeriji“ često su optuživane da su veštice. Predosećaj o lošim stvarima, ako bi se pokazao istinit, sugerisalo bi drugima njihovu veštičju prirodu. Ako biste glasno govorili o svojoj anksioznosti ili imali fizičke simptome koje drugi nisu mogli da objasne na drugi način, dovelo bi do toga da budete „lečeni“ mučenjem (u Španiji), pogubljenjem (u Britaniji) ili spaljivanjem na lomači (uglavnom u Škotskoj). Ironično, ako ste pod stresom zbog toga što je vaša komšinica veštica, to bi moglo da izazove „histeričnu“ reakciju anksioznosti koja bi vas mogla staviti u isti problem. Slično tome, tokom Viktorijanskog doba (1837 - 1901), žene koje su postale „histerične“ smatrane su ludima. Napetost koja je nastala zbog zatvaranja u zatvorenom prostoru bez posla ili bilo čega da se radi dovela je do mnogih takozvanih neobičnih ponašanja. \\
      
             Period koji je značajan za prepoznavanje anksioznih poremećaja kod muškaraca je bio period Američkog gradjanskog rata (1861 - 1865) kada su civili doživljavali simptome kao što su kratak dah i preskakanje srca, srčane palpitacije, ali su tada to stanje prepoznali kao "nervnu slabost" i lečili ga opijumom.  \\
             
            Rusi bili prvi koji su shvatili psihološku prirodu ovog stanja i počeli da šalju psihijatre u rat zajedno sa vojnicima da ih leče nakon bitke tokom rata Rusije sa Japanom 1904. \\

            Većina modernih tehnika za lečenje anksioznosti nastala je posle 1950-ih kao što su: kognitivno-bihevioralna terapija (KBT) razvijena u 1960-im, farmakoterapija, tehnike opuštanja, ekspozicijska terapija. Osim navedenih tehnika, neki od drugih modernih pristupa lečenju anksioznosti uključuju meditaciju, terapiju prihvatanjem i posvećenošću (ACT), kao i terapije uz pomoć virtuelne realnosti koje omogućavaju simulaciju anksioznih situacija radi postepenog izlaganja i prevladavanja straha. 
		
	

		
		\section{Zaključak}	

            Konsultacija sa stručnjakom, kao što je psiholog ili psihijatar, ima ključnu ulogu u pravilnom lečenju i upravljanju anksioznošću. Oni su obučeni da pruže stručni pristup, individualne terapije i korisne tehnike za suočavanje s anksioznošću. Obraćanje stručnom licu je prvi korak koji treba ispuniti kada nas naš um sputava u svakodnevnim aktivnostima i kvari kvalitet života. U prošlosti, anksioznost je bila stigmatizovana i lečena sa ograničenim razumevanjem, dok danas postoji sve veći broj dostupnih resursa i informacija koji pružaju bolje pristupe lečenju. Mentalno zdravlje i anksioznost su postali otvoreno prihvaćene teme, a društvena svest i razumevanje su na najvišem nivou do sada. Ali je važno znati da je svaki put individualan i da jedan pristup nekome može odgovarati, dok nekome ne. Tako da ne treba odustajati u pronalaženju načina da se osećamo što bolje u svojoj koži, i da nam naš um pomaže, a ne odmaže.


		

		
	

 	\newpage

		\addcontentsline{toc}{section}{Literatura}
		\appendix
		
		\iffalse
		\bibliography{seminarski} 
		\bibliographystyle{plain}
		\fi
		
		\begin{thebibliography}{9}
			
			\bibitem{prva}
                \href{https://www.tandfonline.com/doi/full/10.31887/DCNS.2020.22.2/tdienlin?scroll=top&needAccess=true&role=tab&aria-labelledby=full-article}{ Tobias Dienlin, Niklas Johannes, The impact of digital technology use on adolescent well-being, 2022}
			
		
			\bibitem{treca}
                \href{https://www.mentalhealth.org.uk/explore-mental-health/mental-health-statistics/anxiety-statistics}{ Anxiety: statistics}

                \bibitem{druga}
                \href{https://www.ncbi.nlm.nih.gov/pmc/articles/PMC7372668/}{Saba Asif, Azka Mudassar, Talala Zainab Shahzad, Mobeen Raouf, Tehmina Pervaiz, Frequency of depression, anxiety and stress among university students}

                \bibitem{cetvrta}
                \href{https://www.researchgate.net/publication/305345662_Smartphone_gaming_and_frequent_use_pattern_associated_with_smartphone_addiction}{ Liu C. H., Lin S. H., Pan Y. C., & Lin Y. H., Smartphone gaming and frequent use pattern associated with smartphone addiction, 2018 }

                \bibitem{peta}
                \href{ https://truelist.co/blog/social-media-addiction-statistics/}{ Social Media Addiction Statistics – 2023}
	
		
    		\bibitem{sesta}
                \href{ https://www.ncbi.nlm.nih.gov/pmc/articles/PMC2990839/#sec1-1title}{ Hitesh C. Sheth, Zindadil Gandhi, G. K. Vankar Anxiety disorders in ancient Indian literature, 2010}

	        \bibitem{sedma}
                \href{ https://www.calmclinic.com/brief-history-of-anxiety}{  Micah Abraham, A Brief History of Anxiety, 2020
}
                
		\end{thebibliography}
		
		
	\end{document}